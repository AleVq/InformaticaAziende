% !TEX root = ../Informatica e Aziende.tex
\chapter{IT and Business Companies}
\label{chap:IT and Business Companies}
\section{Sistemi Informativi Aziendali}
\label{sec:Business Information System}
Sistemi che supportano le aziende in modo strutturato nei singoli processi di lavoro,
dall'analisi dei dati alla loro elaborazione.
\subsection{Processi Informatizzati 1}
\begin{itemize}
  \item Supporto operativo all'attivit\`a, aumentando l'efficienza del lavoro (mettendo a disposizione
  tutti gli elementi informatici di ricerca e compilazione dei dati), il lavoro diviene pi\`u omogeneo: si
  deresponsabilizza l'operatore, vengono imposti dei vincoli nella raccolta dei dati (migliore
  qualit\`a dei dati), i passi da fare nei processi di lavoro vengono suggeriti dal sistema.
  \item Supporto alla pianificazione dell'azienda, migliore uso delle risorse, chiarezza
  degli obiettivi e dei task gi\`a compiuti: viene data una precisa visione dell'evoluzione
  temporale dell'azienda.
  \item Supporto al controllo: feedback immediati sulle azioni, maggiore tempestivit\`a nel
  rilevare possibili anomalie.
\end{itemize}

\subsection{Processi informatizzati 2}
\label{sub:Processi informatizzati}
L'azienda subisce pressioni da clienti (di grande peso politico nell'azienda, richiedono un canale di controllo diretto),
a livello politico (imposizione
di informatizzazione delle aziende, ad es. Renzi e la fatturazione elettronica),
 concorrenza e opportunit\`a (date dalle evoluzioni tecnologiche).

Nel momento in cui vado a integrare un piccolo, nuovo sistema informativo (ad es. una nuova linea),
i vantaggi saranno piccoli e locali (sistmi locali).
Se un processo informatizzato lega, come un nastro, le entit\`a coinvolte nei flussi di dati, ottengo un vantaggio, in termini complessivi,
abbastanza ampio. Questo porta a una condivisione di conoscenza e produce un flusso informativo
automatizzato (integrazione interna).
Se applico invece nuovi processi informatizzati che sconvolgono altri processi nell'azienda, i vantaggi sono
molto pi\`u evidenti (business process regineering). Porta a un livello di
complessit\`a maggiore, ma anche a migliori prestazioni del sistema.

\subsection{Processi informatizzati 3}
\label{sub:Processi informatizzati 3}
(PULL) Si studia un processo informatizzato per soddisfare un'esigenza propria dell'azienda
(ad es. l'azienda deve gestire talmente tante informazioni che l'Excel non basta pi\`u).\\
(PUSH) Soluzioni spinte dalla tecnologia: l'azienda vede che nel mercato si
presentano nuove soluzioni informative e decide di usarle (es. adozione di un canale e-comm
di un negozio).\\
Esigenza informativa\\
Esigenza per un'azienda di ottenere delle informazioni durante i processi di lavoro.

Nello \textit{schema di Anthony} viene rivisitato lo schema a piramide diviso in sole tre fasce:
\begin{itemize}
  \item Alta direzione
  \item Direzione funzionale
  \item Personale esecutivo
\end{itemize}
Le tre fasce non sono disgiunte: ci possono essere persone che lavorano a vari livelli.\\
Un sistemo informativo si spezza in due parti
\begin{itemize}
  \item sistemi informazionali (supporta le decisioni: analisi)(i strumenti di analisi
sono general purpose)
  \item sistemi operazionali (supporta le attivit\`a:processi, le cui azioni sui dati sono
  simili a quelle del alcolo relazionale).
\end{itemize}


Sono altamente interconnessi: il primo gestisce informazioni a a supporto
dei processi decisionali dai quali, seppur non direttamente, sono prodotti piani
e direttive che regolamentano i sistemi operazionali, il cui scopo \`e la trasformazione
e manipolazione di dati.

\subsubsection{Potenzialit\`a Informatica}

Quanto ha senso l'uso di sistemi informativi lo si pu\`o capire da tre fattori:
\begin{itemize}
  \item Intensit\`a informativa
  \item attrattiva informatica (facilit\`a, reddittivit\`a dell'informatizzazione
  dei processi aziendali)
  \begin{enumerate}
    \item facilit\`a: i modelli di calcolo possono rendere molto facile l'informatizzazione
    di un processo
  \end{enumerate}
  \item Propensione del management all'investimento per l'adozione di SI
\end{itemize}
\subsubsection{Potenzialit\`a informativa}
Quanto intensa \`e la gestione di informazioni nell'azienda?
Parametri:
\begin{itemize}
  \item Complessit\`a dell'azienda (grandezza, diversit\`a dei prodotti etc)
  \item intensit\`a informativa di prodotto
  \item intensit\`a informativa di processo
\end{itemize}
