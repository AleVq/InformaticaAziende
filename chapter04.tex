\chapter{IT and Business Companies}
\label{chap:IT and Business Companies}
\section{Sistemi Informativi Aziendali}
\label{sec:Business Information System}
Sistemi che supportano le aziende in modo strutturato nei singoli processi di lavoro,
dall'analisi dei dati alla loro elaborazione.
\subsection{Processi Informatizzati}
\begin{itemize}
  \item Supporto operativo all'attivit\`a, aumentando l'efficienza del lavoro (mettendo a disposizione
  tutti gli elementi informatici di ricerca e compilazione dei dati), il lavoro diviene pi\`u omogeneo: si
  deresponsabilizza l'operatore, vengono imposti dei vincoli nella raccolta dei dati (migliore
  qualit\`a dei dati), i passi da fare nei processi di lavoro vengono suggeriti dal sistema.
  \item Supporto alla pianificazione dell'azienda, migliore uso delle risorse, chiarezza
  degli obiettivi e dei task gi\`a compiuti: viene data una precisa visione dell'evoluzione
  temporale dell'azienda.
  \item Supporto al controllo: feedback immediati sulle azioni, maggiore tempestivit\`a nel
  rilevare possibili anomalie.
\end{itemize}

\subsection{Processi informatizzati}
\label{sub:Processi informatizzati}
L'azienda subisce pressioni da clienti (di grande peso politico nell'azienda, richiedono un canale di controllo diretto),
a livello politico (imposizione
di informatizzazione delle aziende, ad es. Renzi e la fatturazione elettronica),
 concorrenza e opportunit\`a (date dalle evoluzioni tecnologiche).

Nel momento in cui vado a integrare un piccolo, nuovo sistema informativo (ad es. una nuova linea),
i vantaggi saranno piccoli e locali (sistmi locali).
Se un processo informatizzato lega, come un nastro, le entit\`a coinvolte nei flussi di dati, ottengo un vantaggio, in termini complessivi,
abbastanza ampio. Questo porta a una condivisione di conoscenza e produce un flusso informativo
automatizzato (integrazione interna).
Se applico invece nuovi processi informatizzati che sconvolgono altri processi nell'azienda, i vantaggi sono
molto pi\`u evidenti (business process regineering). Porta a un livello di
complessit\`a maggiore, ma anche a migliori prestazioni del sistema.
