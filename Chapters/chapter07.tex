% !TEX root = ../Informatica e Aziende.tex
\chapter{Sicurezza}
\section{Definizione}
Nello standard \it ISO/IEC 27000 \rm la sicurezza delle informazioni \`e cos\`i definita:\\
\it Information Security: preservation of confidentiality, integrity and availability of information (ISO/IEC 27000).\rm

\section{Vulnerabilit\`a}

L'azienda pu\`o essere vista come un sistema (composto da persone, IT
systems, database) che si interfaccia con cloud, altre aziende, fornitori,clienti.\\
In termini di sicurezza l'azienda ha bisogno di garantire:

\begin{itemize}

\item
  Continuit\`a di servizio
\item
  Uso corretto dell'infrastruttura
\item
  Riservatezza dei dati
\end{itemize}
\section{Fonti di rischio per l'azienda}
\subsection{Eventi non intenzionali}

\begin{itemize}

\item
  Eventi accidentali (naturali), a cui spesso le aziende non pensano
\item
  Interruzioni di servizi esterni (ad es. il SI non \`e utilizzabile
  perch\'e alcuni servizi di terzi a cui l'azienda si appoggia non sono
  temporaneamente disponibili)
\item
  Usure e guasti hardware
\item
  Bug software
\end{itemize}

\subsection{Eventi causati intenzialmente da persone}

\begin{itemize}

\item
  Attacchi indifferenziati (virus, malware, tutte le attivit\`a portate a
  termine da persone scannerizzano la rete in modo ampio, cercano
  vulnerabilit\`a dei sistemi per usare le risorse che riescono a prendere
  in mano..) (indifferenziati perch\'e non sono diretti alla mia azienda)
\item
  Dolo dall'esterno (l'opposto dell'indifferenziato, si vuole attaccare
  proprio quella determinata azienda)
\item
  Rischi dall'interno (determinati da una mancanza di conoscenza, ad es.
  l'utente che usa male il SI, oppure da un dolo, acquisizione di
  informazioni riservate, uso improprio del SI)
\end{itemize}

\subsection{Rischi che nascono dalle cause appena citate}

\begin{itemize}

\item
  Interruzione/rallentamento del lavoro
\item
  Data corruption (dati non pi\'u utilizzabili)
\item
  Perdite economica dirette
\item
  Perdita di competitivit\`a per diffusione di segreti industriali
\item
  Perdita di credibilit\`a (tipo icloud-leaks per la Apple)
\item
  Cause giudiziarie per mancanza di protezione dei dati dei clienti
\end{itemize}

\section{Strategie per migliorare la sicurezza}

\begin{itemize}

\item
  Investimenti in infrastrutture IT: apparati di rete attivi (es. firewall), ridondanza dei dati, software per monitoraggio e controllo
\item
  Investimento in organizzazione: definizioni di vincoli su visibilit\`a dei dati,
  di procedure (a cui tutti gli user del sistema si devono attenere),
  attivit\`a (umane) di monitoraggio e controllo e infine devo nominare
  una o pi\'u persone come responsabili del SI
\item
  Investimento in formazione
\end{itemize}

\`E necessario attivare un'attivit\`a di controllo e nominare un responsabile della sicurezza, ovvero chi si fa carico della responsabilit\`a della sicurezza.

Tutto ci\'o porta dei costi -certi- che non \`e sicuro portino a dei guadagni.\\
Inoltre si aumentano le attivit\`a di controllo, la burocrazia interna, i vincoli che gli utenti del SI devono rispettare, la complessit\`a del sistema.\\
Il rischio pi\`u grande \`e dato dal fatto che molte aziende sono quindi restie a investire somme importanti per la sicurezza.
Cosa spinge un'azienda a investire sulla sicurezza?

\begin{itemize}

\item
  Un manager \emph{consapevole} dei rischi in relazione al ruolo dell'IT
  nell'azienda.
\item
  Obblighi di mercato sulla certificazione della qualit\`a attraverso il
  rispetto di standard di sicurezza quali ISO 27001 e 27002
\item
  Obblighi di legge: trattamento dei dati personali (legge 196/2003), fondamentali ad es. nelle aziende sanitarie
\item
  Obblighi di legge: norme specifiche per ambienti specifici (ad es. nelle istituzioni pubbliche
\end{itemize}
Il trattamento della sicurezza \`e un \emph{processo} aziendale. Non parte
se non vi \`e un impegno forte da parte del management aziendale.\\
La costruzione del sistema di sicurezza comprende la valutazione dei
rischi, del contesto in cui opera l'azienda e la definizione della
politica di sicurezza aziendale, che avviene in funzione delle
precedenti valutazioni.\\
Definita la politica, si passa al progetto vero e proprio. Si
determinano:

\begin{itemize}

\item
  Soluzione per il controllo del rischio (sia esso hardware, software o relativo alla modifica dei processi aziendali)
\item
  Piano di implementazione, che considera i tempi necessari e le persone da coinvolgere
\item
  Responsabilit\`a operative
\item
  Processi di aggiornamento
\item
  Procedure di monitoraggio continuo (appoggiate a software, tipicamente, che
  esegue scansioni continuamente che inviano risultati al gestore della
  sicurezza)
\item
  Procedure di intervento in caso di anomalia
\item
  Audit periodici esterni (ogni 1,2 anni una societ\`a esterna entra in azienda e con uno sguardo obiettivo valuta il sistema)
\item
  Intervallo di revisione (mantiene la complessit\`a dei sistemi bassa, aggiorna gli apparati..)
\end{itemize}

\subsection{Ambiti di tutela}

Vogliamo tutelare l'infrastruttura informatica, i dati, i documenti
informatici, il patrimonio intellettuale.

\subsubsection{Sicurezza dell'infrastruttura}

Vengono usati sistemi di protezione della rete (firewall, antivirus,
sistemi di controllo..) per garantire che l'infrastruttura sia usata in
modo corretto da chi accede con/senza diritto al sistema.\\
Si lavora sulla ridondanza dell'infrastruttura per avere a vari livelli
la disponibilit\`a di qualcosa di sostitutivo (ricambi, backup di dati..).\\
Con questi sistemi non riesco a rilevare usure e guasti. Avverranno
analisi in loco e scansione dell'infrastrutture per scovarli, se
presenti.

\subsubsection{Supporto esterno}

Si definiscono accordi (normativi attraverso contratti ben definiti di
manutenzione e supporto) con partner esterni per il pronto intervento su
software e hardware.

\subsection{Sicurezza dei dati}

Si definiscono delle regole per controllare gli accessi, e si implementano dei
sistemi di tutela dei dati, di certificazione (firme elettroniche, etc.), di
backup dei dati.

\subsection{Cultura della sicurezza}

Bisogna definire un \bf Regolamento Informatico Aziendale\rm: ovvero un documento
redatto dall'azienda e sottoscritto da tutte le persone che operano al suo
interno sui sistemi informatici. Serve a specificare le regole che devono
essere rispettate all'interno dell'azienda con riferimento al settore
informatico: ad esempio nell'uso di PC e dispositivi mobile, della rete internet, della posta elettronica, nell'osservanza delle disposizioni in materia di privacy e cos\`i via.

\section{Linee guida ISO 27001/27002}

L'ISO 27001 definisce delle linee guida di riferimento mentre l'ISO 27002
specifica un elenco di misure di sicurezza. Quest'ultimo indica un approccio
metodologico alla sicurezza aziendale.

Il personale ha bisogno di essere formato, per essere in grado di capire quali
azioni potrebbero portare a un rischio o semplicemente a una diminuzione di
sicurezza.

\subsection{Linee guida}
\subsubsection{Analisi dei rischi}
\begin{itemize}
	\item Definire i requisiti di sicurezza: tali requisiti sono determinati dal contesto in cui l'azienda svolge la sua attivit\`a, dalle normativi, dagli obblighi contrattuali e quindi dalle porzioni del sistema informatico soggette da tali obblighi
	\item Analisi dei rischi
	\item Identificazione delle opzioni per il trattamento dei rischi
	\item Scelta degli obiettivi di controllo per il trattamento dei rischi
	\item Ottenere l’approvazione della direzione circa i rischi residui proposti
\end{itemize}
\subsubsection{Documentare le attivit\`a}
Il sistema aziendale che si occupa della gestione della sicurezza \`e definito nell'ISO 27001 come SGSI: \bf Sistema di gestione della sicurezza informatica\rm.
Un'azienda deve dichiarare:
\begin{itemize}
	\item Le politiche relative all'SGSI
	\item Il campo di applicazione del SGSI
	\item Le procedure e i controlli a supporto del SGSI
	\item Metodologie e rapporto dela valutazione del rischio
	\item Procedure di pianificazione, operativit\`a, misurazione dell'efficacia dei controlli e processi di sicurezza delle informazioni
\end{itemize}
L'azienda deve inoltre registrare interventi sul sistema ed eventi rilevanti, ad es. un breach nel sistema.
\subsubsection{Formare il personale, attribuire responsabilit\`a}
L'azienda deve identificare le responsabilit\`a del personale relative al SGSI.\\
Coloro che sono responsabili del SGSI deve avere le competenze per poter svolgere i suoi compiti: devono quindi essere previsti corsi di formazione, valutazione e registrazione di corsi, addstramenti, abilit\`a, esperienze e qualifiche del personale.
\subsubsection{Revisione periodica}
Il SGSI deve essere periodicamente revisionati per assicurarne l'idoneit\`a. La valutazione del SGSI deve includere l'identificazione dei cambiamenti necessari e le possibili opportunit\`a di miglioramento.
\section{Linee guida sulla continuit\`a operativa nelle PA}
Consistono in un documento estremamente regolato che norma le pratiche relative alla Digitalizzazione dei processi della Pubbliche Amministrazioni italiane.\\
Tale documento non tratta solo di rischi relativi all’infrastruttura informatica, ma anche dell’infrastruttura che la sostiene: i sistemi elettrico, antincendio, antintrusione, cos\`i come le strutture e l'ubicazione dei data center.\\
Il documento fornisce anche strumenti di autovalutazione, per identificare il livello di continuit\`a atteso per ogni servizio.
\subsection{Autovalutazione}
Sono fornite tre direttrici di analisi:
\begin{itemize}
	\item Servizio\\
			In base alla tipologia, numerosit\`a e criticit\`a dei servizi erogati, si valutano i potenziali danni per l'organizzazione o per i suoi utenti in caso di mancata erogazione.\\
			Ad es.: tipi di utenti e dati trattati, frequenza di erogazione del servizio, possibilit\`a di recuperare i dati (da un sistema di backup ad esempio)..
	\item Organizzazione\\
			In base alla complessit\`a amministrativa e strutturale dell'organizzazione si suggerisce un determinato dimensionamento delle soluzioni tecnologiche da adottare\\
			Ad es.: numero di sedi, di responsabili del trattamento dei dati, dei trattamenti censiti, degli addetti all'erogazione dei servizi..
	\item Tecnologia\\
			In base alle dimensioni e complessit\`a del fattore tecnologico si suggeriscono determinate tipologie e natura delle soluzioni da adottare\\
			Ad es.: presenza di un dipartimento IT, numerosit\`a degli addetti IT, numero di server, di postazioni di lavoro, di archivi usati..
\end{itemize}