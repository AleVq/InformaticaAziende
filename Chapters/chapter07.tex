% !TEX root = ../Informatica e Aziende.tex
\chapter{Sicurezza}

\section{Vulnerabilit\`a}

L'azienda pu\'o essere vista come un sistema (composto da persone, IT
systems, DB) che si interfaccia con cloud, altre aziende, fornitori,
clienti.\\
In termini di sicurezza l'azienda ha bisogno di garantire

\begin{itemize}

\item
  Continuit\`a di servizio
\item
  Uso corretto dell'infrastruttura
\item
  Riservatezza dei dati
\end{itemize}

\subsection{Fonti di rischio per l'azienda}

\begin{itemize}

\item
  Eventi accidentali (naturali), a cui spesso le aziende non pensano
\item
  Interruzioni di servizi esterni (ad es. il mio SI non \`e utilizzabile
  perch\'e alcuni servizi di terzi a cui l'azienda si appoggia non sono
  temporaneamente disponibili)
\item
  Usure e guasti HW
\item
  Bug SW
\end{itemize}

\subsection{Vi sono anche eventi causati intenzialmente da persone}

\begin{itemize}

\item
  Attacchi indifferenziati (virus, malware, tutte le attivit\`a portate a
  termine da persone scannerizzano la rete in modo ampio, cercano
  vulnerabilit\`a dei sistemi per usare le risorse che riescono a prendere
  in mano..) (indifferenziati perch\'e non sono diretti alla mia azienda)
\item
  Dolo dall'esterno (l'opposto dell'indifferenziato, si vuole attaccare
  proprio quella determinata azienda)
\item
  Rischi dall'interno (determinati da una mancanza di conoscenza, ad es.
  l'utente che usa male il SI, oppure da un dolo, acquisizione di
  informazioni riservate, uso improprio del SI)
\end{itemize}

\subsection{Rischi che nascono dalle cause appena citate}

\begin{itemize}

\item
  Interruzione/rallentamento del lavoro
\item
  Data corruption (dati non pi\'u utilizzabili)
\item
  Perdite economica dirette
\item
  Perdita di competitivit\`a per diffusione di segreti industriali
\item
  Perdita di credibilit\`a (tipo icloud-leaks per la Apple)
\item
  Cause giudiziarie per mancanza di protezione dei dati dei clienti
\end{itemize}

\subsection{Strategie per migliorare la
sicurezza}

\begin{itemize}

\item
  Investimenti in infrastrutture IT: apparati di rete attivi (es.
  firewall), ridondanza dei dati, SW per monitoraggio e controllo
\item
  Investimento in organizzazione: def. vincoli su visibilit\`a dei dati,
  di procedure (a cui tutti gli user del sistema si devono attenere),
  attivit\`a (umane) di monitoraggio e controllo e infine devo nominare
  una o pi\'u persone come responsabili del SI
\item
  Investimento in formazione
\end{itemize}

C'è da attivare un'attivit\`a di controllo e nominare un responsabile della 
sicurezza, ovvero chi si fa carico della responsabilità della sicurezza.

Tutto ci\'o porta dei costi -certi- che non \`e sicuro portino a dei
guadagni.\\
Inoltre si aumentano le attivit\`a di controllo, la burocrazia interna, i
vincoli che gli utenti del SI devono rispettare, la complessit\`a del
sistema.\\
Molte aziende sono restie a investire somme importanti per la
sicurezza.

Cosa spinge un'azienda a investire sulla sicurezza?

\begin{itemize}

\item
  Un manager \emph{consapevole} dei rischi in relazione al ruolo dell'IT
  nell'azienda.\\
\item
  Obblighi di mercato sulla certificazione della qualit\`a attraverso il
  rispetto di standard di sicurezza quali ISO 27001 e 27002
\item
  Obblighi di legge: trattamento dei dati personali (legge 196/2003),
  fondamentali ad es. nelle aziende sanitarie
\item
  Obblighi di legge: norme specifiche per ambienti specifici (ad es.
  nelle istituzioni pubbliche
\end{itemize}

Il trattamento della sicurezza \`e un \emph{processo} aziendale. Non parte
se non vi \`e un impegno forte da parte del management aziendale.\\
La costruzione del sistema di sicurezza comprende la valutazione dei
rischi, del contesto in cui opera l'azienda e la definizione della
politica di sicurezza aziendale, che avviene in funzione delle
precedenti valutazioni.\\
Definita la politica, si passa al progetto vero e proprio. Si
determinano:

\begin{itemize}

\item
  soluzione per il controllo del rischio
\item
  Piano di implementazione
\item
  Responsabilit\`a operative
\item
  Processi di aggiornamento
\item
  Procedure di monitoraggio continuo (appoggiate a SW, tipicamente, che
  esegue scansioni continuamente che inviano risultati al gestore della
  sicurezza)
\item
  Procedure di intervento in caso di anomalia
\item
  Audit periodici esterni (ogni 1,2 anni una societ\`a esterna che entra
  in azienda e con uno sguardo obiettivo valuta il sistema)
\item
  Intervallo di revisione (mantiene la complessit\`a dei sistemi bassa,
  aggiorna gli apparati..)
\end{itemize}

\subsection{Ambiti di tutela}

Vogliamo tutelare l'infrastruttura informatica, i dati, i documenti
informatici, il patrimonio intellettuale.

\subsubsection{Sicurezza dell'infrastruttura}

Vengono usati sistemi di protezione della rete (firewall, antivirus,
sistemi di controllo..) per garantire che l'infrastruttura sia usata in
modo corretto da chi accede con/senza diritto al sistema.

Si lavora sulla ridondanza dell'infrastruttura per avere a vari livelli
la disponibilit\`a di qualcosa di sostitutivo (ricambi, backup di dati..).

Con questi sistemi non riesco a rilevare usure e guasti. Avverranno
analisi in loco e scansione dell'infrastrutture per scovarli, se
presenti.

\subsubsection{Supporto esterno}

Si definiscono accordi (normativi attraverso contratti ben definiti di
manutenzione e supporto) con partner esterni per il pronto intervento su
SW e HW.

\subsection{Sicurezza dei dati}

Si definiscono delle regole per controllare gli accessi, e si implementano dei
sistemi di tutela dei dati, di certificazione (firme elettroniche, etc.), di
backup dei dati.

\subsection{Cultura della sicurezza}

Bisogna definire un Regolamento Informatico Aziendale: ovvero un documento 
redatto dall'azienda e sottoscritto da tutte le persone che operano al suo 
interno sui sistemi informatici. Serve a specificare le regole che devono 
essere rispettate all'interno dell'azienda con riferimento al settore 
informatico.

\subsection{Linee guida ISO 27001/27002}

L'ISO 27001 definisce delle linee guida di riferimento mentre l'ISO 27002
specifica un elenco di misure di sicurezza. Quest ultimo indica un approccio 
metodologico alla sicurezza aziendale.

Il personale ha bisogno di essere formato, per essere in grado di capire quali
azioni potrebbero portare a un rischio o semplicemente a una diminuzione di 
sicurezza.

% !TEX root = ../Informatica e Aziende.tex
\chapter{Etica}

\section{Gestire i dilemmi etici in azienda}
Bisogna puntare all'autoregolazione tramite strumenti sociali: dalla
formazione, definizione di macropolitiche (linee guida dettata
dall'amministrazione centrale dell'azienda la cui valenza \`e alla stregua
di una normativa nell'azienda), definizioni di certificazione da parte
delle aziende in tema etico (che possano fondare degli standard),
sensibilizzazione sociale (ha un carattere molto forte, ad es. tramite
pressioni sociali)\\
Nell'azienda vengono assegnate assunzioni di responsabilit\`a sulle
conseguenze delle proprie azioni. Intorno alle decisioni ci si pone dei
processi di tipo etico. Il problema etico implica un dilemma
(riconoscere che la conseguenza delle mie azioni pu\'o avere risvolti
negativi su altri) di cui si ricerca una soluzione ottimale attraverso
determinati passi:

\begin{itemize}

\item
  Identificato il dilemma, si vogliono trovare gli elementi e i fatti
  che lo circondano.\\
\item
  Si definiscono i punti di conflitto e si identificano i valori di
  ordine superiore che diano un "cappello" di legittimit\`a sociale del
  proprio agire (quali sono i valori che sostengono la mia azienda? come
  fa il pubblico a considerare eticamente corretta la mia azienda).\\
\item
  Si identificano poi gli interessi, coloro che sono coinvolti nel mio
  dilemma etico\\
\item
  Identificazione delle opzioni che si possono adottare per contenere le
  potenziali conseguenze
\end{itemize}

Per facilitare la conduzione di una analisi etica, vi sono vari
principi:

\begin{itemize}

\item
  Mettersi nei panni degli altri
\item
  Correttezza: un'azione non corretta anche solo per uno allora non \`e
  corretta per nessuno (bisogna quindi focalizzarsi anche sui casi
  singoli)
\item
  Ripetibilit\`a: un'azione che non si pu\'o ripetere non va fatta
\item
  Massimizzazione del valore (monetario ma anche sociale, relazionale,
  che non sono tangibili o calcolabili)
\item
  Minimizzare i rischi per aziende o persone
\item
  Propriet\`a: "non esiste un pranzo gratis" (qualcuno ci ha lavorato
  dietro e vale anche per beni immateriali come il software)
\end{itemize}

Aspetti toccati nell'ICT:

\begin{itemize}

\item
  Privacy
\item
  Propriet\`a intellettuale (segreti commerciali in qualit\`a di
  "gentlement agreement" tra aziende, copyright come legge che si
  occupa di copie illecite di prodotti materiali o non, brevetti come
  forma pi\'u forte di tutela della propriet\`a intellettuale che include un
  impegno economico non indifferente e in campo del software prevedono
  iter molto lunghi, usati come strumenti di tutela della propriet\`a
  intellettuale)
\item
  Qualit\`a dei SI (qualit\`a dati ed errori dei SI)
\item
  Responsabilit\`a sociale e giuridica
\end{itemize}

Problemi di qualit\`a della vita

\begin{itemize}

\item
  Rapidit\`a dei cambiamenti
\item
  vedi slides
\end{itemize}

\subsection{Documenti in azienda}

Vi sono vari motivi per cui spingere a dematerializzare i documenti in
un'azienda, dall'aumento della quantit\`a, alla circolazione dei
documenti, alla necessit\`a di aumentare le prestazioni aziendali
(facilit\`a nel trovare un documento archiviato ad es. vs presenza di
copie inutili e maggiore vulnerabilit\`a, oltre che consumi di carta e
spazio quando si usa il cartaceo), a spinte esterne (politica).\\
I sistemi di gestione documentale hanno diverse finalit\`a:

\begin{itemize}

\item
  reperimento di file efficiente e tramite paramentri di ricerca
  flessibili
\item
  disponibilit\`a di primitive per interfacciarsi coi sistemi operazionali
\item
  facilit\`a l'uso di versioni diverse di un documento (poter trattare le
  variazioni nella stesura di un documento)
\item
  fornire garanzie sulla conformit\`a del documento
\item
  eliminare sprechi di carta
\end{itemize}

Il sistema di gestione documentale \`e a s\'e stante rispetto agli ERP e
prevede un'archiviazione sostitutiva che \`e un metodo pi\'u oneroso per la
gestione dei documenti, poich\'e appesantisce di molto il flusso
documentale.

\subsubsection{Documento informatico}

Ha un'informazione strutturata relativi a riferimenti di archiviazione
(controlli, trattamento dei dati) e uso (metadati).\\
La parte vera e propria del documento pu\'o esistere (file) o meno, poich\'e
possono essere memorizzati solo metadati relativi al documento e quelli
lo rappresentano, o esistono solo riferimenti alla posizione della copia
cartacea. Nella maggior parte dei casi esiste uno o pi\'u file (ad es.
manuale con tutte le immagini che vi appaiono).

Caratteristiche del documento informatico

\begin{itemize}

\item
  Informazioni strutturali: (che dipendono dal file che viene
  archiviato): formato e dimensione del file. Il formato detta la
  possibilit\`a di accesso al file dalle postazioni di lavoro, la
  dimensione discrimina la possibilit\`a di trasferimento di file in posti
  remoti.
\item
  Informazioni di processo: relative all'ambito applicativo e al tempo
  (chi ha archiviato il documento, da che postazione, stato attuale del
  file nel sistema: approvato, in fase di revisione, obsoleto..);
  informazioni legate all'ambito di applicabilit\`a, classe di
  appartenenza del documento che determina l'ambito applicativo in cui
  pu\'o essere usato, privilegi di accesso, caratteristiche semantiche
\item
  Informazioni semantiche: attribuite al documento riflettono il
  contenuto di quell'istanza del documento: data, mittente, scadenza,
  oggetto del documento che posso divenire criteri di ricerca di quel
  documento. Sono proprio informazioni di contenuto. Classi distinte di
  documenti hanno caratteristiche semantiche distinte.
\end{itemize}

\subsubsection{Funzionalit\`a di trattamento del
documento}

Archivio

\begin{itemize}

\item
  Processo di archiviazione: messa a disposizione di una funzione
  interattiva, trattamento di masse di documenti in modo automatizzato
  (elaborati e caricati in modo massivo, automatico), possibilit\`a di
  acquisizione dei dati tramite scanner
\item
  Ricerca: sostenute dai metadati o ricerche full-text per trovare
  parole chiave presenti all'interno del file
\item
  Modifica dei metadati: sempre disponibile, ma pu\'o essere limitata alle
  persone con privilegi specifici
\item
  Eliminazione: dei metadati e del file, operazione delicata: potrebbe
  creare incosistenze sui sistemi ERP che si riferiscono al file. È
  un'operazione subordinata ai permessi
\end{itemize}

Lavoro sui contenuti

\begin{itemize}

\item
  Accesso statico dei contenuti: apertura, visualizzazione, preview
  (operazioni date dalle primitive del sistema operativo)
\item
  Inoltro dei documenti: avviene tramite mail, PEC, fax
\item
  Autenticazione: usate marche temporali per porre sigilli temporali
  sulle modifiche di un file
\item
  Modifica del file: usualmente asincrona (non \`e detto che software
  gestionali supportino operazioni su certi tipi di file)
\item
  Versionamento: ha senso su documenti che prevedono un iter nel tempo
  di cui voglio mantenere traccia dei passi (ad es. un manuale tecnico,
  perch\'e \`e sulle versioni che baso manutenzioni e altro, la domanda \`e
  "quali classi di documenti necessitano del versionamento?")
\end{itemize}

\subsubsection{Uso nei flussi di lavoro}

Come il documento vinee usato nei flussi di lavoro dell'azienda.\\
La funzione propria del sistema di gestione documentale \`e quella in cui
l'azienda prevede un processo in cui prende atto di tutti i documenti
uscenti ed entranti.

In alcune aziende sono previsti protocolli di ingresso/uscita. Si
vogliono registrare in questi casi informazioni quali mittente,
destinatario, data e ID (informazioni acquisite), canali di
comunicazione (fax, mail, PEC). Se il protocollo \`e in ingresso, si
attivano dei flussi documentali: archiviazione del documento (pu\'o non
essere fatta da chi ha in carico il protocollo, perch\'e non tenuto a
conoscere il formato o la sintassi del documento) e quello di
distribuzione (che potrebbero generare dei passi ∂i archiviazione
successivi).\\
Questa funzione di protocollazione diviene molto delicati in caso di PA
(si introducono certificazioni che appesantiscono il protocollo).

\subsubsection{Flussi documentali}

Per loro natura, i documenti hanno un proprio flusso (dalla creazione e
revisione all'approvazione del documento ai flussi di lavoro delle
persone che gestiscono e modificano il documento).

Si possono definire dei grafi condizionali che rappresentano i flussi.

\subsubsection{Sistemi di
certificazione}

Generano enormi quantit\`a di dati e flussi documentali. Sono soggetti a
versionamento, a marcaturale temporale (ad es. password cambiata ogni 30
giorni come previsto dal protocollo aziendale), integrazione a flussi di
approvazione (ricerca e rimozione di anomalie ed errori ad es.). Il
sistema pu\'o essere integrato oppure appoggiato al sistema documentale.

\subsubsection{Flussi operativi}

Integrare documenti all'interno dei flussi operativi (che trattano
informazioni strutturate di norma). Si possono risolvere situazioni
ambigue risalendo ai documenti originali ad es. Nei flussi operativi
posso usare documenti generati esternamente: fatture, disegni CAD, ad
es. nei processi di manutenzione le macchine soggette a manutenzione,
quando al tecnico preposto ad eseguirle riceve tutti i documenti
relativi alle macchine, affinch\'e il suo lavoro divenga pi\'u incisivo,
perch\'e non si richiede che vada a spulciare tra scartoffie non
strutturate.

Le funzioni dei docomenti nei flussi operativi sono: archiviazione,
refernza, accesso, processo. I documenti sono prodotti a partire da dati
procedurali (ordini, fatture..), sono sempre datati e possono essere
prodotti singolarmente o massivamente.\\
I documenti sono datati, per rimarcare la loro validit\`a in uno specifico
intervallo di tempo. I riferimenti ai documenti possono essere portati
sulle anagrafiche, ad es. archivio di un utente e fattura emanata a suo
carico, mappe, funzioni di supporto.

\subsubsection{Conservazione}

Implica che io posso conserare documenti che devono essere mantenuti per
legge per anni in archivi elettronici che devono essere strutturati in
maniera particolare affinch\'e i dati possono essere estratti in un certo
modo: i documenti devono poter essere considerati certi e legali in
termini fiscali e le copie estratte devono garantire conformit\`a
all'originale.\\
Gi\`a nel 1994 si discuteva di conservazione sostitutiva: si poteva
conservare i documenti su dispositivi ottici, ma tale sistema non \`e mai
decollato.\\
Nel 2004 si ridefinita la procedura di archiviazione che trascendeva i
dispositivi ottici. La PM nello stesso anno ha assunto che un documento
potesse essere "originale" nella sua forma elettronica, non
necessariamente nella sua forma stampata.\\
Dal 2005 si lavora per definire un quadro normativo che possa sostenere
da archivi di carta a archivi digitali.\\
Nel 2015 \`e stato normato l'obbligo per le PA di adottare il protocollo
(di ingresso/uscita) informatizzato.

\subsubsection{Caratteristiche richieste nella conservazione
sostitutiva}

\begin{itemize}
\item
  Autenticit\`a: \`e prevista una figura di riferimento che certifica che il
  documento che viene creato \`e conforme all'originale (firme
  elettroniche e marche temporali garantiscono l'autenticit\`a)
\item
  Integrit\`a: il documento dev'essere sempre estratto in modo conforme
  all'originale
\item
  Affidabilit\`a del sistema
\item
  Leggibilit\`a: deve essere garantita la leggibilit\`a dei documenti anche
  a distanza di tempo
\item
  Reperibilit\`a: i documenti devono essere facilmente reperibili
\item
  Definizione del responsabile della conservazione: si fa carico di
  gestione e aggiornamento del processo di conservazione, delle attivit\`a
  correnti, del monitoraggio sulle corretta funzionalit\`a del sistema e
  di supporto ai pubblici ufficiali/organismi che devono analizzare i
  documenti Oggetto della conservazione sostitutiva Viene trattato il
  pacchetto informativo: oggetto composito formato da metadati e tutti i
  file che formano il documento logico. I metadati sono distinti da
  quelli dell'ERP. Inoltre abbiamo:
\item
  Pacchetto di versamento: come i documenti vengono archiviati
\item
  Pacchetto di archiviazione: \`e il pacchetto di versamento addizionato
  di firme elettroniche, marcature temporali, impronte docuemntali
\item
  Pacchetto di distribuzione: pacchetto informativo inviato dal sistema
  di conservazione all'utente in risposta a una sua richiesta
\end{itemize}

L'unico scopo della conservazione sostitutiva \`e la conservazione e
archiviazione delle copie certe dei documenti richiesti per legge
(contabili, relativi allo stock).

La conservazione sostitutiva avviene o in casa o outsource, per non
appesantire le aziende (specie quelle piccole). Allora la responsabilit\`a
della conservazione sostitutiva va all'azienda outsourcing.

Conservazione sostitutiva e sistemi di gestione operativa documentali
sono sempre distinti, seppure integrati (di norma solo per azioni di
versamento, per evitare aggiornamenti sporchi).

\end{document}
