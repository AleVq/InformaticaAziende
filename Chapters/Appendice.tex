% !TEX root = ../Informatica e Aziende.tex

% Questa è la lezione dell'avvocato?
% Se si io la metterei alla fine di tutto come "cose aggiuntive".
%Quando aggiungerò altre lezioni, le posizionerò prima delle "lezioni supplementari".
% PS. Perché mi sostituisci alcuni doppi apici con due backtick??

\chapter{Lezioni supplementari}

\section{Pubblica amministrazione}

\subsection{PEC}

Raccomandata con ricevuta di ritorno in forma digitale.
\#\#Autenticazione di un soggetto Fondamentale in diritto
amministrativo: l'identità digitale. Si passa dall'uso delle credenziali
alla firma digitale (gestione chiavi di un ente certificatore
riconosciuto dallo stato, legate a documenti come carta d'identità) (mi
identifico rispetto a un file). \#\#SPID È una identità digitale. Idea:
credenziali universali (in termini di enti della PA ma anche per enti
privati). Identity provider: ente che rilascia mail e passoftwareord usando la
tua carta d'identità. Il certificato di firma scade dopo 3 anni dalla
data di emissione. Firme digitali:

\begin{itemize}
\item
  Cades, contenuta in una busta elettronica crittografata
\item
  Pades (accordo 2009 dello stato con adobe), embedded nel pdf
\item
  Xades, firma dell'XML, simile al Pades
\end{itemize}

Regolamento eIDAS.

\subsection{Conservazione digitale dei
documenti}

Processo di archiviazione al quale il legislatore ha dato valore legale.
Problema: valore legale di un documento digitale archiviato. I sistemi
di conservazione oramai sono servizi che si comprano da aziende
specializzate. Si usano tracciati XML per salvare metadati relativi ai
file. Viene usata la firma elettronica del responsabile della
conservazione. (pcm dicembre 2013)

\subsection{Firma elettronica}

Processo informatica che permette di identificare un soggetto. Quattro
tipi:

\begin{itemize}
\item
  Firma elettronica debole: credenziali, ha minimo valore legale
\item
  Firma elettronica avanzata: secondo sistema di firma, che sulla base
  di un processo collegato a un documento, permette di identificarne la
  persona che lo ha firmato: ad es. il tablettino in banca: firma
  grafometrica (dato biometrico, basato su velocità, pressione etc per
  identificare la firma sul tablettino)
\end{itemize}

\section{Aziende Private}

\subsection{Privacy by design}

Progettare i prodotti software in maniera tale che vengano prevenuti
rischi per la privacy degli utenti (ad es. data breach).
