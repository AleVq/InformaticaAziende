% !TEX root = ../Informatica e Aziende.tex
\chapter{Sistemi ERP}
\label{chap:Sistemi ERP}
\section{Sistemi Tecnici}
\label{sec:Sistemi Tecnici}

I sistemi di questo tipo sono fortemente verticalizzati: vengono
per lo pi\`u prodotti ad hoc, in ambiti specifici.
\subsection{Commesse d'impianto}
\label{sub:Commesse d'impianto}
Uno dei sistemi tecnici pi\`u complessi sono quelli che gestiscono
le commesse d'impianto. Ci sono molte aziende che producono prodotti che
non sono altro che progetti che vengono portati alla produzione (es. Danieli,
che progetta e produce impianti molto complessi per la metallurgia, venduti e
prodotti su commisione).\\
Nelle commesse devo avere un sistema che supporti i diversi piani
per ogni parte del processo (dalla progettazione all'implementazione).
Nelle commesse inoltre la parte di pagamento ha una cadenza scandita dai suoi stati
d'avanzamento: pu\`o durare mesi, anni e i pagamenti possono essere
effettuati con cadenze prefissate.\\
\subsection{Scheduling di produzione}
\label{sub:Scheduling di produzione}
Dato un ordine, voglio poter elaborare dei dati per riuscire a dire
al cliente quanto tempo ci metto a effettuare la consegna. Tali dati possono essere:
\begin{itemize}
  \item Giacenza (del prodotto in magazzino), allocazioni (ordini gi\`a effettuati dai clienti).
  Se giacenza - allocazioni < 0, allora devo produrre l'oggetto richiesto
  dal cliente: allora i tempi si allungano e dipendono da:
  \begin{enumerate}
    \item tempo, macchine, persone
    \item tempo di attrezzaggio della macchina
    \item tempo di passaggio da una fase all'altra della produzione
  \end{enumerate}.
  Nel caso in cui non abbia le materie prime per la produzione, devo considerare
  anche i tempi di approvvigionamento del fornitore, dovendo considerare quindi
  anche le distanze azienda-fornitore, la cardinalit\`a dell'ordine di approvvigionamento.
\end{itemize}
Lo scheduling di produzione sono utili quando le aziende voglio tenere una piena occupazione
di macchine e personale, di modo che tali risorse (che non sono finite) non rimangano ferme.
I pianificatori di produzione vanno a ottimizzare l'uso di risorse umane e impiantistiche in funzione
dell'attuale situazione dell'azienda. Sono software che effettuano dei calcoli molto complessi.
\subsection{Sistemi CAD}
\label{sub:Sistemi CAD}
Computer Aided Design.\\
Ha varie estensioni, tra cui CAE, CAM e CIM (Computer Integrated Manufacturing).
\section{Sistemi di ufficio e organizzazione}
\label{sec:Sistemi di ufficio e organizzazione}
\begin{itemize}
  \item Automazione di ufficio
  \item Gestione elettronica dei documenti, un plusvalore al sistema che supporta il lavoro quotidiano
  \item Document flow (flusso dei documenti tra le varie parti del processo) e Workflow
  \item Strumenti per il lavoro collaborativo
\end{itemize}
