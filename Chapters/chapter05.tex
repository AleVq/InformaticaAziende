% !TEX root = ../Informatica e Aziende.tex
\chapter{Sistemi ERP}
\label{chap:Sistemi ERP}
\section{Sistemi Tecnici}
\label{sec:Sistemi Tecnici}

\subsection{Enterprise Resource Planning}
\label{sub:Enterprise Resource Planning}
Sistemi altamente configurabili che permettono un'ampia informatizzazione
dell'azienda in tutte le aree. Un sistema ERP \`e gestito da un solo fornitore,
\`e altamente configurabile e fornisce una gestione di dati di alt\`a qualit\`a
(nei termini visti precedentemente). Permettono di integrare delle personalizzazioni,
costruendo estensioni del sistema.\\
Il sistema ERP \`e disaccoppiato dal sistema aziendale (ne trascende la struttura).\\
\subsubsection{ERP per l'amministrazione}
\label{subs:ERP per l'amministrazione}
A partire dalla contabilit\`a ordinaria, a seconda del tipo di azienda, l'ERP integra moduli
relativi alla contabilit\`a analitica (registrazione ingressi e uscite delle singole parti dell'azienda),
contabilit\`a IVA (solo per aziende private), budget, controllo di gestione, cespiti e cos\`i via.\\
I diversi moduli presentano interazioni intrinseche con altri moduli. \\
L'ERP \`e usato per trattare determinate informazioni nei vari flussi:
\begin{itemize}
	\item Logistica (movimentazione dei materiali nell'azienda)
	\item Vendite (trattamento delle informazioni commerciali e relative ai clienti)
	\item Acquisti (informazioni sui rapporti con i fornitori)
	\item Produzione (informazioni sulla struttura dei prodotti, delle risorse necessarie
	alla produzioni, dei processi produttivi)
\end{itemize}
Vi sono esigenze informative delle aziende che gli ERP non possono soddisfare.\\
Controllo di qualit\`a, associata a un onere gestionale corposo, il cui volume aumenta
l'attrattiva informatica.\\
Nel settore di ricerca e sviluppo non si seguono processi standardizzabili o definibili
a priori, per cui si possono trovare solo prodotti specializzati come simulatori o CAD,
un minimo supporto informatico gestionale (per monitorare i costi), strumenti di avanzamento
del progetto, gestione della documentazione relativa al progetto e al prodotto. Questo settore
\`e molto destrutturato.\\
Nella manutenzione degli impianti, vi sono sistemi cosiddetti scadenziari (relativi alla pianificazione
degli interventi), sistemi relativi al trattamento dei protocolli e alla gestione dei registri.\\
Poco gettonati invece sono i sistemi relativi alla tesoreria (informazioni
sui rapporti con istituti di credito). Alcuni istituti possono essere pi\`u convenienti di altri,
in base a numerosi parametri e ad algoritmi ad elevata complessit\`a: usati per di pi\`u dalle grandi aziende.\\
Vi sono processi che ampliano le funzionalit\`a di base dell'ERP (ERP II,
che amplia il raggio d'azione dell'ERP portandolo al di fuori dell'azienda). Alcuni esempi:
\begin{itemize}
	\item CRM (Customer Relationship Management), sostiene tutti i contatti azienda-cliente (specie nella relazione diretta richiesta dal cliente
	come il processo di supporto ai clienti) tra i cui scopi vi \`e la fidelizzazione dei clienti
	nel post-vendita. Ha un'architettura di supporto diversa da quella dell'ERP (ad es. tramite web interface, o mobile app per gli utenti),
	per questo \`e separato dall'ERP.
	\item Supply chain management (SCM), sostiene il rapporto col fornitore tramite internet,
	abbattendo costi e tempi della fornitura.
	\item e-commerce: interfaccia utente che permette di eliminare una funzione aziendale, si integra con l'ERP
	principalmente nel flusso attivo e nella logistica (che pu\`o avere una complessit\`a piuttosto elevata).
	Il sistema si divide in B2C (business to customer) o B2B (business to business).
\end{itemize}

I sistemi ERP (Enterprise Resource Planning) sono fortemente verticalizzati: vengono
per lo pi\`u prodotti ad hoc, in ambiti specifici.
\subsection{Commesse d'impianto}
\label{sub:Commesse d'impianto}
Alcuni dei sistemi tecnici pi\`u complessi sono quelli che gestiscono
le commesse d'impianto. Ci sono molte aziende che producono prodotti che
non sono altro che progetti che vengono portati alla produzione (es. Danieli,
che progetta e produce impianti molto complessi per la metallurgia, venduti e
prodotti su commisione).\\
Nelle commesse devo avere un sistema che supporti i diversi piani
per ogni parte del processo (dalla progettazione all'implementazione).
Nelle commesse inoltre la parte di pagamento ha una cadenza scandita dai suoi stati
d'avanzamento: pu\`o durare mesi, anni e i pagamenti possono essere
effettuati con cadenze prefissate.\\
\subsection{Scheduling di produzione}
\label{sub:Scheduling di produzione}
Dato un ordine, voglio poter elaborare dei dati per riuscire a comunicare al cliente i tempi di attesa previsti. Tali dati possono essere:
\begin{itemize}
  \item Giacenza (del prodotto in magazzino), allocazioni (ordini gi\`a effettuati dai clienti).
  Se 
  $$
   giacenza - allocazioni < 0
  $$
  allora non sono presenti sufficienti scorte in stock e si rende necessario produrre nuovi lotti dell'oggetto richiesto dal cliente: a questo punto, l'allungamento dei tempi dipende da:
  \begin{enumerate}
    \item tempo, macchine, persone
    \item tempo di attrezzaggio della macchina
    \item tempo di passaggio da una fase all'altra della produzione.
  \end{enumerate}
  \item
  Nel caso in cui non siano disponibili le materie prime per la produzione, si devono considerare anche i tempi di approvvigionamento del fornitore, dovendo considerare quindi anche le distanze azienda-fornitore e la cardinalit\`a dell'ordine di approvvigionamento.
\end{itemize}
Lo scheduling di produzione \`e utile quando le aziende vogliono tenere una piena occupazione
di macchine e personale, di modo che tali risorse (che non sono infinite) non rimangano ferme.\\
I pianificatori di produzione vanno a ottimizzare l'uso di risorse umane e impiantistiche in funzione
dell'attuale situazione dell'azienda. Sono software che effettuano dei calcoli molto complessi.
\subsection{Sistemi CAD}
\label{sub:Sistemi CAD}
Computer Aided Design.\\
Ha varie estensioni, tra cui CAE, CAM e CIM (Computer Integrated Manufacturing).
\section{Sistemi di ufficio e organizzazione}
\label{sec:Sistemi di ufficio e organizzazione}
Esistono vari tipi di SI che hanno lo scopo di migliorare il workflow negli uffici. Alcuni di questi verranno approfonditi pi\`u avanti.
\begin{itemize}
  \item Automazione di ufficio
  \item Gestione elettronica dei documenti, un plusvalore al sistema che supporta il lavoro quotidiano
  \item Document flow (flusso dei documenti tra le varie parti del processo) e Workflow
  \item Strumenti per il lavoro collaborativo
\end{itemize}
