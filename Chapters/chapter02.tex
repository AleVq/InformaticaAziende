% !TEX root = ../Informatica e Aziende.tex
\chapter{IT in Business Companies}
\subsection{Sui dati e le informazioni}
Definiamo \textit{dato} un flusso grezzo di fatti accadduti all'interno dell'azienda o nel suo ambiente.
L'\textit{informazione} \`e invece un dato che viene modellato in una forma tale da essere utile e significativo per le persone.\\
Un sistema informatico prende in input i dati e li trasforma in informazioni.

\section{Introduzione}
Tutto ci\`o che permette di raccogliere, registrare, memorizzare e condividere le informazione va a formare il sistema informativo nelle aziende.\\
Definiamo \textbf{sistema informativo} l'insieme dei mezzi tecnici, delle procedure organizzative, delle risorse umane il tutto finalizzato alla gestione delle informazioni prodotte.
Un sistema informativo detta e influenza le capacit\`a di un'azienda di lavorare.
\'E sempre in evoluzione, nonostante le aziende non siano portate ai cambiamenti.\\


Diverse aziende hanno un punto in comune, rispetto ai SI: adottano, chi pi\`u chi meno, sistemi informatici a supporto dei sistemi aziendali.
Diverse invece sono le differenze:
\begin{itemize}
\item Dominio/specializzazione dell'azienda
\item Copertura/ampiezza: fattori economici e di cultura dell'azienda influenzano il numero di processi aziendali che vengono supportati da strumenti informatici.
\item Livello di informatizzazione, determinato da quanto pu\`o un SI essere d'aiuto a un'azienda.
\end{itemize}

In un'azienda le informazioni fluiscono sia formalmente (supportate in modo chiaro dal SI) che informalmente (non necessariamente supportate dal SI). Una comunicazione informale pu\`o divenire formale se sostenuta dal SI (includendo metadati e ulteriori informazioni). Gran parte della comunicazione nelle aziende \`e informale.

\section{Impatto dell'ICT sull'azienda}
\subsection{Impatto economico}
Facile da misurare (investimento immediato vs ammortamento), l'impatto economico si traduce con
\begin{itemize}
\item riduzione dei costi delle transazioni
\item riduzione delle dimensioni dell'azienda (in termini di risorse umane) grazie all'automazione che, tra l'altro, aumenta la quantit\`a di merce prodotta mantenendo lo stesso livello di qualit\`a.
\item riduzione della necessit\`a della gestione intermedia (responsabili di terzo, quarto livello), che si occupa di coordinare gruppi di persone operative comunicando comandi dai livelli superiori a quelli inferiori e riportando problemi e aggiornamenti dai livelli inferiori a quelli superiori. Si arriva pertanto a un appiattimento della gerarchia aziendale.
\end{itemize}

\subsection{Impatto nell'organizzazione}
\begin{itemize}
\item Si passa da una struttura piramidale dell'azienda a una pi\`u flessibile.
\item Nuovi modelli di business: aziende virtuali che trattano esclusivamente informazioni (senza sede fisica, materiali o catene di produzione).
\item Aumento della flessibilit\`a (adattabile ai cambiamenti costanti del mercato) e resilienza (adattabilit\`a a cambiamenti improvvisi) dell'azienda.
\end{itemize}

\subsection{Impatto sui processi decisionali}
Difficile da misurare, i side-effect non sono evidenti.
\begin{itemize}
\item Fornita maggiore disponibilit\`a dei dati organizzati in base ai processi decisionali
\item Semplificazione delle decisioni prese a pi\`u livelli gerarchici
\end{itemize}
