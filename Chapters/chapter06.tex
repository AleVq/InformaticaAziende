% !TEX root = ../Informatica e Aziende.tex
\chapter{Business e management}
\label{chap:Business e management}

\section{Sistemi informazionali}
\label{sec:Sistemi informazionali}
I decisori dell'azienda (ad es. managers, ma talvolta anche stakeholders o altri) richiedono dati strutturati ed elaborati.
Sono richiesti quindi un database (con tutte le specifiche che abbiamo gi\`a visto) e degli strumenti di analisi dei dati, che si possono dividere in tre tipologie:
\begin{itemize}
  \item Reporting: producono liste statiche che rappresentano la situazione aziendale in un certo momento; sono strumenti veloci, il cui punto forte \`e l'invarianza nella forma di quello che rappresentano
  \item Sistemi di analisi interattiva guidata da ipotesi: di fronte a eventi imprevisti, il decisore parte dai report
  e cerca anomalie che giustifichino l'evento imprevisto formulando una certa ipotesi che guida
  le sue ricerche nei dati
  \item Sistemi di data mining: una funzione di analisi dei dati che, elaborando i dati compresi in un certo intervallo
  di tempo, espone le anomalie che riscontra; il loro punto di forza sta nel fatto che possono trovare anomalie che
  il decisore non avrebbe trovato (\`e una tecnologia piuttosto costosa)
\end{itemize}
Tali tipologie categorie sono presentate in ordine di complessit\`a crescente.

\subsection{Caratteristiche dei sistemi informazionali}
\label{sub:Caratteristiche dei sistemi informazionali}
\subsubsection{Finalit\`a}
Il sistema informazionale:
\begin{itemize}
  \item descrive il passato dell'azienda
  \begin{enumerate}
    \item storicit\`a: (pu\`o arrivare fino a 5, massimo 10, anni)
    \item dettaglio: i dati sono presentati in forma aggregata per evitare inutili
    livelli di dettaglio; sono disponibili diversi livelli di aggregazione, per variare
    il livello di dettaglio dei dati presentati
    \item accesso ai dati: in sola lettura (nel supporto ai processi decisionali),
    gli aggiornamenti vengono fatti solo in determinati periodi di tempo
  \end{enumerate}
  \item "prevede" il futuro tramite elaborazione dei dati
\end{itemize}
\subsection{Modello multidimensionale}
\label{sub:Modello multidimensionale}
Nei sistemi di analisi ai fini di supporto alle decisioni, il modello dei dati adottato
\`e di tipo multidimensionale: le informazioni sono articolate intorno a soggetti, fatti/eventi e misure.
Ciascuno dei tre elementi appena citati sono in interazione reciproca con cardinali\`a 1-1 o anche 1-n.
Lo spazio delle informazioni pu\`o essere visto come insieme di matrici multidimensionali.
Spesso la matrice \`e denormalizzata. Si privilegia una struttura piana nell'organizzazione dei dati,
per consentire delle sintesi dei dati pi\`u veloci.\\
Si viene a produrre un ipercubo. Il singolo elemento $A_{i,j,..}$ \`e un fatto elementare a cui viene associato un valore
numerico che lo quantifica (misura).
L'iperspazio nell'ipercubo \`e poco popolato, perch\`e i clienti non acquistano tutti i giorni, a tutte le ore, a tutti i minuti, in tutti negozi o le sedi e cos\`i via.\\
Le caratteristiche numeriche elaborate dal modello multidimensionale
tendono a non avere ambiguit\`a.\\
I fatti elementari da cui estrapoliamo dati quantitativi possono
comprendere diverse misure. I dati estrapolati vanno poi aggregati
tramite operatori quali somma, media, max, min..\\
La dimensione invece rappresenta una delle coordinate dell'iperspazio
nell'iper\-cubo. Il dominio della dimensione dev'essere finito. Esempi di domini possono essere il
numero dei clienti (nell'ordine delle migliaia tipicamente), il numero
delle merci (decine di migliaia, tipicamente)..\\
Le gerarchie (o attributi dimensionali) sono insiemi di attributi
collegati ad una dimensione, ad es. alla data sono associati ora,
giorno, mese, anno. Allo stesso modo per i clienti: settori in cui
operano, luogo in cui si trovano, tipologia.. Pi\`u ricche sono tali
gerarchie, pi\`u interessanti diventano le aggregazioni di questi dati.

\subsection{Caratteristiche
strutturali}\label{caratteristiche-strutturali}

\begin{itemize}

\item
  Multidimensionalit\`a
\item
  Granularit\`a\\
  Grado minimo di aggregazione dei dati, corrispondente a un fatto
  elementare
\item
  Arco temporale\\
  Intervallo temporale (fino ad alcuni anni) coperto dai dati raccolti
\end{itemize}

\subsection{Caratteristiche
funzionali}\label{caratteristiche-funzionali}

\begin{itemize}

\item
  Integrazione dei dati intorno al concetto dei soggetti\\
  I dati sono organizzati intorno a determinati eventi concettuali che
  si sono verificati in azienda
\item
  Accessibilit\`a\\
  I dati devono essere facilmente visibili, fruibili. La capacit\`a di
  aggregare i dati del sistema dev'essere ottimizzata per mantenere
  veloce la consultazione dei dati
\item
  Flessibilit\`a e sintesi\\
  Lo spazio multidimensionale deve ottemperare alle aspettative degli
  utenti (ad es. relative alle dimensioni dei dati)
\item
  Correttezza\\
  I dati devono essere corretti
\item
  Completezza\\
  Il sistema deve essere in grado di rappresentare tutti gli eventi
  rilevanti per le analisi
\end{itemize}

\subsection{Data warehouse}\label{data-warehouse}

Letteralmente, magazzino dei dati. Vi sono memorizzati tutti i fatti e
le informazioni che possono essere utili all'azienda e le relative alle analisi.
Pu\`o raggiungere dimensioni notevoli. I dati sono "raw", completi e consistenti.\\
Nell'architettura dei sistemi di data warehousing ogni elemento
d'archivio \`e legato a un determinato processo. Data warehouse fa parte
dei dati informazio\- nali. Negli strumenti di analisi dei dati compaiono
data mining e OLAP (online analytical processing). Alcuni database integrano il sistema OLAP e vengono chiamati MOLAP.

\begin{itemize}

\item
  Vantaggi: le query anche pi\`u complesse sono eseguite velocemente
\item
  Svantaggi: occupano troppo spazio (la matrice multidimensionale \`e
  istanziata nella sua totalit\`a) e non esiste alcuno standard
\end{itemize}

Alcuni database relazionali integrano la struttura multidimensionale (ROLAP)
realizzando un modello semi-relazionale.

\begin{itemize}

\item
  Vantaggi: spazio occupato minimo (sono memorizzati solo eventi
  verificati, la matrice non \`e istanziata in toto)
\item
  Svantaggi: scarsa efficienza nell'esecuzione delle query di
  aggregazione e una pi\`u bassa capacit\`a di risposta
\end{itemize}

Nella vita reale si viene a usare un approccio ibrido (HOLAP): il data
warehouse \`e implementato su base relazionale mentre il data mart su base
multidimensionale.

\subsection{Data Mart}\label{data-mart}

\`E un sottospazio tematico del data warehouse.
Contiene una porzione dei dati della data warehouse, aggregati e
preparati per il singolo utente. Se l'utente \`e a livello di direzione
centrale la panoramica comprender\`a tutti i dati, ma il livello di
aggregazione sar\`a ben superiore a quello barebone della data warehouse.

\subsection{Ciclo di vita del data warehouse}\label{ciclo-di-vita-del-data-warehouse}

La costruzione avviene tramite un processo iterativo. Vi sono analisi
delle sorgenti (dei dati), progettazione concettuale, logica e
implementativa.

Il popolamento dei dati avviene tramite estrazione dei dati dalle
sorgenti. Tali dati sono poi integrati e trasformati per conformarsi
alla consistenza dei dati della warehouse. Sono poi puliti, riconoscendo
ed eliminando errori e incongruenze. I dati sono poi usati per popolare
la warehouse.

\subsection{Analisi OLAP}\label{analisi-olap}

Navigazione interattiva sui dati multidimensionali.\\
\`E l'equivalente per l'analisi dei dati del browser per internet (difatti si possono inserire bookmark, viene tenuta una history dei passi fatti). Passi:

\begin{itemize}

\item
  drill down (passo a una vista pi\`u dettagliata)
\item
  roll up (passo a una vista pi\`u aggregata)
\item
  slice (visualizzazione 2d di una porzione della matrice
  multidimensionale)
\item
  dice (filtro i fatti elementari per identificare un sottospazio
  dell'iperspazio)
\item
  pivot (cambio del punto di vista)
\end{itemize}

\subsubsection{Limiti di OLAP}\label{limiti-di-olap}

\begin{itemize}

\item
  Le informazioni non sono facilmente identificabili
\item
  Insufficiente perch\`e troppo dipendente dai processi deduttivi dei decisori
\end{itemize}

\subsection{Data mining}\label{data-mining}

Identifica tutti quei sistemi che sono in grado di applicare delle
procedure di analisi dei dati in grado di far emergere delle condizioni notevoli, rilevanti. Passi del processo di mining:

\begin{itemize}

\item
  pulizia
\item
  integrazione
\item
  selezione
\item
  trasformazione
\item
  data mining
\item
  valutazione
\item
  presentazione dati
\end{itemize}

Si attiva una procedura di scansione dei dati alla ricerca di pattern.
Molti pattern potrebbero essere irrilevanti: gli utenti devono
specificare al sistema la direzione in cui "scavare"  (mining).

Abbiamo 5 funzioni di mining:

\begin{itemize}

\item
  caratterizzazione e discriminazione (i dati sono aggregati per
  caratteristiche)
\item
  analisi associativa (quali sono le condizioni che si verificano
  contemporaneamente quando si verifica un certo evento, ad es. Walmart
  aveva trovato che una vendita di pannolini per bambini era associata a
  una vendita di birra, allora Walmart ha provato ad avvicinare negli
  scaffali pannollini e birra e questo ha portato a una maggiore vendita
  di birre)
\item
  classificazione e predizione (rilevo delle informazioni relative a un
  certo soggetto, in base a queste posso calcolare la probabilit\`a che
  tale soggetto assuma comportamenti anomali, ad es. nei preventivi per
  le assicurazioni auto)
\item
  analisi dei cluster (cerca i luoghi in cui si addensa l'informazione, ad es. classi omogenee di soggetti: tipi di prodotti che vengono comprati
  prevalentemente da un tipo di cliente)
\item
  analisi degli outlier (cerca i luoghi in cui si verifica la
  rarefazione dell'informazione, cerco gli elementi che deviano dalle situazioni standard: frodi e altro)
\end{itemize}

\subsection{Campi di applicazione di data mining e
warehousing}\label{campi-di-applicazione-di-data-mining-e-warehousing}

\begin{itemize}

\item
  Analisi finanziarie
\item
  Analisi marketing
\item
  Analisi vendite
\end{itemize}

Fonti di informazioni ulteriori:

\begin{itemize}

\item
  dati di accesso al sito dell'azienda
\item
  log delle attivit\`a sulla rete
\end{itemize}

\section{Gestione della conoscenza}\label{gestione-della-conoscenza}

Aiuta a esplicitare a livello organizzativo la conoscenza dell'azienda.\\
Include le tecniche viste fin'ora, che sono solo quantitative. Altri strumenti sono:

\begin{itemize}

\item
  document management, per la conservazione di altri tipi di conoscenza,
  non necessariamente quantitativa
\item
  portali, motori di ricerca per la diffusione della conoscenza
\end{itemize}

A livello di organizzazione esiste una cultura della conoscenza (idea
che la conoscenza \`e dappertutto nell'azienda, che \`e preziosa e aiuta a
migliorare l'azienda, vale la pena diffonderla all'interno
dell'organizzazione, formare il personale..)
