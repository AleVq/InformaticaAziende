% !TEX root = ../Informatica e Aziende.tex
\chapter{Business e management}
\label{chap:Business e management}

\section{Sistemi informazionale}
\label{sec:Sistemi informazionale}
I decisori dell'aziendale (managers etc) richiedono dati strutturati ed elaborati.
Sono richiesti quindi un database (con tutte le specifiche che abbiamo gi\`a visto) e degli strumenti di analisi dei dati, che si possono dividere in tre tipologie:
\begin{itemize}
  \item reporting: producono liste statiche che rappresentano la situazione aziendale in un certo momento; sono strumenti veloci, il cui punto forte \`e invarianza nella forma di quello che rappresentano
  \item sistemi di analisi interattiva guidata da ipotesi: di fronte a eventi imprevisti, il decisore parte dai report
  e cerca anomalie che giustifichino l'evento imprevisto formulando una certa ipotesa che guida
  le sue ricerche nei dati
  \item sistemi di data mining: una funzione di analisi dei dati che, elaborando i dati compresi in un certo intervallo
  di tempo, espone le anomalie che riscontra; il loro punto di forza sta nel fatto che possono trovare anomalie che
  il decisore non avrebbe trovato. \`E una tecnologia piuttosto costosa
\end{itemize}
Queste tre categorie sono di complessit\`a crescente, partendo dalla prima.

\subsection{Caratteristiche dei sistemi informazionali}
\label{sub:Caratteristiche dei sistemi informazionali}
Finalit\`a
\begin{itemize}
  \item descrive il passato dell'azienda
  \begin{enumerate}
    \item storicit\`a: (pu\`o arrivare fino a 5, massimo 10, anni)
    \item dettaglio: i dati sono presentati in forma aggregata per evitare inutili
    livelli di dettaglio; sono disponibili diversi livelli di aggregazione, per variare
    il livello di dettaglio dei dati presentati
    \item accessoai dati: in sola lettura (nel supporto ai processi decisionali),
    gli aggiornamenti vengono fatti solo in determinati periodi di tempo
  \end{enumerate}
  \item "prevede" il futuro tramite elaborazione dei dati
\end{itemize}
\subsection{Modello multidimensionale}
\label{sub:Modello multidimensionale}
Nei sistemi di analisi ai fini di supporto alle decisioni, il modello dei dati adottato
\`e di tipo multidimensionale: le informazioni sono articolate intorno a soggetti, fatti/eventi e misure.
Ciascuno dei tre elementi appena citati sono in interazione reciproca con cardinali\`a 1-1 o anche 1-n.
Lo spazio delle informazioni pu\`o essere visto come insieme di matrici multidimensionali.
Spesso la matrice \`e denormalizzata. Si privilegia una struttura piana nell'organizzazione dei dati,
per consentire delle sintesi dei dati pi\`u veloci.\\
Si viene a produrre un ipercubo. L'elemento $A_{i,j,..}$ \`e un fatto elementare a cui viene associato un valore
numero che lo quantifica (misura).
L'iperspazio nell'ipercubo \`e poco popolato, perch\`e i clienti non acquistano tutti i giorni etc..
