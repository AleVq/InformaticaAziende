% !TEX root = ../Informatica e Aziende.tex
\chapter{Cambiamenti}

\section{Spinte al cambiamento}

Il cambiamento-innovazione \`e spinto da:

\begin{itemize}
  \item
    Contesto politico/sociale (normative e finanziamenti)
  \item
    Pu\`o essere indotta dal contesto del mercato (standard di fornitura o attese
    del mercato)
  \item
    Pu\`o essere suggerito dall'evoluzione tecnologica (nuova tecnologia che da
    nuove possibilit\`a)
\end{itemize}


\subsection{Dove sta l'informatico?}

All'interno dell'azienda l'informatico deve rilevare le esigenze e coinvolgere competenze
interne e/o esterne. Il suo ruolo \`e di Referente per il SI, e CED
(Responsabile, Progettuale, Operativo).

All'esterno dell'azienda \`e un consulente (Aiuta l'azienda a convergere nelle
azioni di cambiamento, porta conoscenze, competenze e punto di vista esterno,
Tecnologia $<-->$ Organizzazione). Deve fare il \textbf{fornitore di soluzioni}
tecnologiche, dove realizza una parte del cambiamento in accordo con azienda
e - se presente - consulente.

\section{Cambiamento}

\begin{itemize}
  \item
  Continuo (manutentivo, omeostatico)

  \begin{itemize}
    \item Obiettivo: Garantire il funzionamento del sistema
    \begin{itemize}
      \item
      Mantenere coerenza
      \item
      Favorire adeguamenti in linea con i processi esistenti
    \end{itemize}
    \item
    Azioni
    \begin{itemize}
      \item Monitoraggio, controllo, pianificazione di attivit\`a ricorrenti...
    \end{itemize}
    \item
    Problemi
    \begin{itemize}
      \item Potrebbe essere attivit\`a invisibile
    \end{itemize}
  \end{itemize}

  \item
  Discontinuo (evolutivo)

  \begin{itemize}
    \item
    Obiettivo: Rinnovare/ottimizzare il sistema
    \begin{itemize}
      \item
      Ricostruire la coerenza persa
      \item
      Ottimizzare il lavoro e ridurre i costi
      \item
      Supportare nuovi servizi
    \end{itemize}

    \item
     Azioni
     \begin{itemize}
       \item
       Progetti di rinnovamento trasversale
       \item
       Informatizzazione di processi esistenti
       \item
       Introduzione di nuovi processi informatizzati
     \end{itemize}

     \item
     Problemi
     \begin{itemize}
       \item
       Difficolt\`a di convergenza
       \item
       Distanza tra attesa e risultati
       \item
       Resistenze al cambiamento
     \end{itemize}
  \end{itemize}
\end{itemize}


\subsection{Favorire il cambiamento}

L'informatico dovrebbe avere delle conoscenze (tecniche, di dominio, contesto),
deve essere in grado di definire il problema, l'oggetto, i ruoli e i modi.
Inoltre deve essere in grado di condurre il progetto, il gruppo e le riunioni.

In quanto gestori di sistemi informativi, il nostro scopo \`e favorire il
cambiamento nell'azienda, inteso come rinnovamento e innovazione delle
infrastruttura. Vediamo ora i passi per favorire i cambiamenti nei SI aziendali.

\subsection{Conoscere la tecnica}
Nelle aziende si suppone che l'informatico sappia fare di tutto
(dall'usare Excel a gestire database, alla gestione della
sicurezza e cos\`i via). Queste aspettative sono irragionevoli. \\
Chi ricopre la funzione di informatico in azienda non pu\`o che fare da
mediatore in grado di ricondurre dei problemi riportati in azienda a qualche
esperto esterno che funga da risolutore: \`e necessario quindi saper riconoscere
la qualit\`a dei possibili fornitori o agenti esterni ma anche la loro
pertinenza all'ambito in cui ci collochiamo.
L'informatico pu\`o anche non conoscere certi tipi di problemi, ma deve comunque
essere in grado di informarsi autonomamente e di intervenire in prima battuta.
L'azienda deve sostenere l'informatico.

\subsection{Conoscere il dominio}
Definiamo \textit{dominio} come l'ambito di competenza dell'azienda con la quale
il manager di SI si trova a interloquire. \\
Aziende che lavorano negli stessi ambiti presentano simili peculiarit\`a  nei
problemi che si presentano, seppur declinati in modi diversi. \\
Ogni dominio deve sottostare a regole o normative vigenti (ad esempio la
privacy) dettate da attese e regole che si d\`a il mercato. Per condurre degli
interventi efficaci dobbiamo conoscere il dominio in cui l'azienda si muove, i
% Cosa intendi per committenti?
vincoli del nostro agire e comunicare coi committenti in un linguaggio a loro
comprensibile, poich\'e non \`e detto che conoscano il linguaggio
tecnico-informatico.
I nostri commitenti possiedono un linguaggio formato da acronimi e convenzioni
di  significato su parole comuni: capire tali sfumature aiuta a identificare
problemi e soluzioni per l'azienda. \\
Gli informatici interni all'azienda assimilano tali convenzioni interne, essendo
immersi nell'ambiente aziendale. Gli esterni invece sanno poco o nulla del
contesto aziendale, quindi devono farsi una panoramica abbastanza grossolana
(ma che sia efficace) di conoscenze del dominio: attraverso interviste col
committente, ad esempio. Uno dei must in questi casi \`e chiedere chiarimenti,
anche banali, in caso di dubbi sul dominio. Le interviste vanno  affiancate da
ricerche fatte in ambiti esterni all'azienda:  quest'ultima ha una sua visione
sui suoi domini, visione che potrebbe  essere imprecisa (un caso paradigmatico
riguarda un ente di protezione civile che aveva alcuni riferimenti normativi ad
obbligo non noti all'interno dell'azienda, si \`e dovuto quindi rivedere tutto
il sistema in funzione della normativa, piuttosto che della sola  efficacia ed
efficienza dell'azienda).

\subsection{Conoscere il contesto}

Oltre al contesto esplicito comunicato dal  personale aziendale, ne esiste uno
implicito che pu\`o essere pi\`u o meno diverso, che  narra della struttura
informale dell'azienda, come si muovono i flussi d'informazioni: vi \`e
un'articolazione fatta di persone che influenza la struttura gerarchica
dell'azienda (ad esempio un responsabile  commerciale che influenza i processi
aziendali pi\`u dell'amministratore  delegato: ha pi\`u senso parlare col primo
che non col secondo per  informarsi sul contesto dell'azienda, sulle culture
personali, sui punti nevralgici di decisione nell'azienda). \\
La  lettura del contesto \`e sempre filtrata dalle proprie culture, dai
pregiudizi, dagli schemi mentali e dalle aspettative che ne derivano: \`e necessario
guardare con uno sguardo aperto agli ambienti aziendali, essendo consapevoli
dei propri filtri percettivi. Tale consapevolezza permette di capire che
l'interlocutore  pu\`o avere una visione pi\`u o meno distante dalla mia
(possono cio\`e  esserci punti di vista diversi che possono essere conciliati).\\
Conoscere il contesto \`e necessario, poich\'e il cambiamento pu\`o essere
spesso vissuto come negativo da parte di determinate aziende (in cui  ad
esempio i lavoratori sono abituati a determinate routine) e pu\`o  risultare
impegnativo e frustrante per le persone, ad esempio nel caso in cui debbano
apprendere l'uso di nuove strumentazioni o software: conoscere il  contesto
aiuta in questo senso.\\

Con contesto esplicito intendiamo i dettagli dell'azienda (quante persone,
quanti dipartimenti, relazioni etc).

Con contesto implicito specifichiamo la struttura informale dell'azienda,
ovvero in che modo vengono prese le decisioni.

Se il contesto esplicito \`e lineare, quello implicito \`e contraddittorio e
ambiguo, difficile da cogliere.

\subsection{Immaginiamo il contesto}

Teniamo in considerazione che in questa tabella per startup intendiamo l'idea
bella e positiva di ci\`o che \`e, e per azienda storica intendiamo un'azienda locale
di media grandezza.\\
La lettura del contesto \`e sempre filtrata dalle proprie culture. Ognuno vede ci\`o
che riconosce.\\
C'\`e da sviluppare una certa consapevolezza per sapere quali sono le mie mappe
di riferimento. La mia lettura sar\`a diversa da quella del mio interlocutore.
\\
{
\centering
\begin{tabularx}{\textwidth}{| Y | Y | Y | Y | Y | Y |}
	\hline
	Azienda & Startup & Azienda storica & Cooperativa sociale & Banca & GDO evoluta \\
	\hline
	Articolazione                & Piccola e destrutturata                                                   & Mediamente strutturata e struttura mediamente leggibile & Tendenzialmente sciolta             & Ampia, rigida, formale, burocratica                               & Strutturata per efficienza \\
	\hline
	Valori                       & Creare innovazione / impatto, molto centrati sul prodotto, e condivisione & Identit\`a, qualit\`a, responsabilit\`a sociale               & Robin Hood - Responsabilit\`a sociale & Economici                                                         & Velocit\`a ed efficienza     \\
	\hline
	Modello di conduzione        & Partecipato                                                               & Gerarchico padronale                                    &                                     & Burocratici, spersonalizzato                                      & Basato su obiettivi        \\
	\hline
	Coinvolgimento delle persone & Coinvolti e parzialmente responsabili                                     & Affettivo                                               & Coinvolgimento rispetto ai principi & Coinvolgimento rispetto all'interazione, ma anche spersonalizzato & Nullo\\
	\hline
\end{tabularx}
}


\subsection{Cambiamenti a logica lineare}

Nel mondo ingegneristico vige l'idea che i progetti debbano avere un andamento
lineare (il processo produttivo ha un punto di partenza, un percorso, un arrivo,
tutti e tre predeterminati). Se ci sono cause di problemi, sono considerate
fattori esterni, che vanno gestiti. \\
In termini di cambiamenti, facciamo una distinzione tra due tipologie distinte:
\begin{itemize}
  \item Determinati: sono responsabilit\`a dell'azienda, previsti dai piani lineari
  \item Subiti: vengono dall'esterno e sono quindi responsabilit\`a altrui
\end{itemize}
Nel mondo ingegneristico si assume che gli individui siano guidati da
razionalit\`a e le persone, gruppi e organizzazione sono visti come variabili
dipendenti dal sistema: una persona pu\`o essere sostituita da un'altra per uno
stesso incarico senza considerare possibili differenze di rendimento. \\
In generale i processi aziendali non sono mai lineari: vi possono essere intoppi
in un qualunque processo. Tali intoppi fanno emergere possibili imprevisti che
tuttavia possono essere visti come opportunit\`a per cambiare modi e
destinazioni dei processi lavorativi. \\
Nell'azienda vi possono essere alcuni lavoratori che non rispondono bene ai
cambiamenti: \`e quindi necessario prendere atto che tale cambiamento suscita
una determinata reazione nel personale, piuttosto che targare come inadeguate
tali persone senza considerare gli effetti che i cambiamenti possono avere sulle
persone. \\
Inoltre, a dispetto della visione lineare dei processi aziendali, cambiare una
persona in un gruppo (ad esempio in una business unit) significa stravolgere le
dinamiche sociali del gruppo. Il cambiamento si attua attravero le persone, e 
influenza la cultura delle persone.

Il cambiamento non \`e necessariamente positivo, \`e difficile e suscita
passioni.

Qualsiasi cambiamento provoca relazioni negative.
