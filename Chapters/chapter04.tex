% !TEX root = ../Informatica e Aziende.tex
\chapter{IT and Business Companies}
\label{chap:IT and Business Companies}
\section{Sistemi Informativi Aziendali}
\label{sec:Business Information System}
Sistemi che supportano le aziende in modo strutturato nei singoli processi di lavoro,
dall'analisi dei dati alla loro elaborazione.
\subsection{Processi Informatizzati 1}
\begin{itemize}
  \item Supporto operativo all'attivit\`a, aumentando l'efficienza del lavoro (mettendo a disposizione
  tutti gli elementi informatici di ricerca e compilazione dei dati), il lavoro diviene pi\`u omogeneo: si
  deresponsabilizza l'operatore, vengono imposti dei vincoli nella raccolta dei dati (migliore
  qualit\`a dei dati), i passi da fare nei processi di lavoro vengono suggeriti dal sistema.
  \item Supporto alla pianificazione dell'azienda, migliore uso delle risorse, chiarezza
  degli obiettivi e dei task gi\`a compiuti: viene data una precisa visione dell'evoluzione
  temporale dell'azienda.
  \item Supporto al controllo: feedback immediati sulle azioni, maggiore tempestivit\`a nel
  rilevare possibili anomalie.
\end{itemize}

\subsection{Processi informatizzati 2}
\label{sub:Processi informatizzati}
L'azienda subisce pressioni da clienti (di grande peso politico nell'azienda, richiedono un canale di controllo diretto),
a livello politico (imposizione
di informatizzazione delle aziende, ad es. Renzi e la fatturazione elettronica),
 concorrenza e opportunit\`a (date dalle evoluzioni tecnologiche).

Nel momento in cui vado a integrare un piccolo, nuovo sistema informativo (ad es. una nuova linea),
i vantaggi saranno piccoli e locali (sistmi locali).
Se un processo informatizzato lega, come un nastro, le entit\`a coinvolte nei flussi di dati, ottengo un vantaggio, in termini complessivi,
abbastanza ampio. Questo porta a una condivisione di conoscenza e produce un flusso informativo
automatizzato (integrazione interna).
Se applico invece nuovi processi informatizzati che sconvolgono altri processi nell'azienda, i vantaggi sono
molto pi\`u evidenti (business process regineering). Porta a un livello di
complessit\`a maggiore, ma anche a migliori prestazioni del sistema.

\subsection{Processi informatizzati 3}
\label{sub:Processi informatizzati 3}
(PULL) Si studia un processo informatizzato per soddisfare un'esigenza propria dell'azienda
(ad es. l'azienda deve gestire talmente tante informazioni che l'Excel non basta pi\`u).\\
(PUSH) Soluzioni spinte dalla tecnologia: l'azienda vede che nel mercato si
presentano nuove soluzioni informative e decide di usarle (es. adozione di un canale e-comm
di un negozio).\\
Esigenza informativa\\
Esigenza per un'azienda di ottenere delle informazioni durante i processi di lavoro.

Nello \textit{schema di Anthony} viene rivisitato lo schema a piramide diviso in sole tre fasce:
\begin{itemize}
  \item Alta direzione
  \item Direzione funzionale
  \item Personale esecutivo
\end{itemize}
Le tre fasce non sono disgiunte: ci possono essere persone che lavorano a vari livelli.\\
Un sistemo informativo si spezza in due parti
\begin{itemize}
  \item sistemi informazionali (supporta le decisioni: analisi)(i strumenti di analisi
sono general purpose)
  \item sistemi operazionali (supporta le attivit\`a:processi, le cui azioni sui dati sono
  simili a quelle del alcolo relazionale).
\end{itemize}


Sono altamente interconnessi: il primo gestisce informazioni a a supporto
dei processi decisionali dai quali, seppur non direttamente, sono prodotti piani
e direttive che regolamentano i sistemi operazionali, il cui scopo \`e la trasformazione
e manipolazione di dati.

\subsubsection{Potenzialit\`a Informatica}

Quanto ha senso l'uso di sistemi informativi lo si pu\`o capire da tre fattori:
\begin{itemize}
  \item Intensit\`a informativa
  \item attrattiva informatica (facilit\`a, reddittivit\`a dell'informatizzazione
  dei processi aziendali)
  \begin{enumerate}
    \item facilit\`a: i modelli di calcolo possono rendere molto facile l'informatizzazione
    di un processo
  \end{enumerate}
  \item Propensione del management all'investimento per l'adozione di SI
\end{itemize}
\subsubsection{Potenzialit\`a informativa}
Quanto intensa \`e la gestione di informazioni nell'azienda?
Parametri:
\begin{itemize}
  \item Complessit\`a dell'azienda (grandezza, diversit\`a dei prodotti etc)
  \item intensit\`a informativa di prodotto
  \item intensit\`a informativa di processo
\end{itemize}
\subsubsection{Attrattiva informatica}
\label{subs:Attrattiva informatica}
Fattori che determinano l'attrattiva informatica per un processo
\begin{itemize}
  \item Grado di strutturazione: pi\`u il processo \`e struttato pi\`u \`e facile
  informatizzarlo (in soldoni, se \`e facile trovare un algoritmo che rappresenti un processo, sar\`a
  facile trovare un SI a supporto di tale processo)
  \item Complessit\`a: pi\`u un processo \`e complesso, pi\`u difficile sar\`a
  la sua informatizzazione
  \item Ripetitivit\`a: pi\`u un processo viene ripetuto seguendo sempre
  gli stessi passi, pi\`u una sua informatizzazione sar\`a auspicata
  (ad es. nella pianificazione dei budget la raccolta dei dati da consuntivare e la loro elaborazione
  possono essere automatizzate, per\`o la sua bassa frequenza di ripetizione e la sua variabilit\`a
  ne rende utile l'informatizzazione solo nelle aziende pi\`u grandi)
  \item Volume: maggiore il numero di dati da elaborare in un processo, maggiore
  l'attrattiva informatica
\end{itemize}

\subsection{Composizione dei sistemi operazionali}
\label{sub:Composizione dei sistemi operazionali}
Nella catena del valore di Porter, le attivit\`a primarie, essendo quelle che
trovano un riscontro immediato in termini di profitti e dal punto di vista dei clienti
(per questo le aziende sono incentivate a investirci).
Le attivit\`a primarie esprimono la specificit\`a dell'azienda, sono
pertinenti all'azienda in questione: difficile trovare un supporto operativo off-the-shelves.\\
Al contrario, le attivit\`a di supporto (gestione risorse umane che include anche l'istruzione
del personale, spese di gestione ad es. degli uffici..) richiedono investimenti che non
restituiscono direttamente un guadagno.
La catena di Porter fornisce con le attivit\`a di supporto punti comuni per tutte le aziende (la gestione
delle risorse umane \`e parte di tutte le aziende), per i quali sono previste
standardizzazioni.
\subsection{Informazione operativa}
\label{sub:Informazione operativa}
Virtualmente, i clienti percepiscono come unitario l'archivio consultabile
dell'azienda (DB relazionale ad es.).\\
La qualit\`a dei dati \`e regolamentata dalle norme ISO (8402-1984).
\subsubsection{Caratteristiche strutturali}
\label{subs:Caratteristiche strutturali}
\begin{itemize}
  \item Aggregazione: posso avere un grado di sintesi maggiore o minore
  (i dati raccolti su un determinato processo sono pi\`u o meno aggregati,
  cio\`e ottenuti tramite elaborazione di dati analitici)
  \item Tempificazione: arco temporale cui il dato si riferisce
  \item Dimensionalit\`a: numero di parametri necessari per accedere
  a un'informazione
\end{itemize}
\subsubsection{Caratteristiche funzionali}
\label{subs:Caratteristiche funzionali}
Parametri che fanno s\`i che l'informazione raccolta tramite SI sia di scarsa
o di buona qualit\`a.
\begin{itemize}
  \item Correttezza: pi\`u l'informazione \`e controllata pi\`u \`e corretta
  (ad es. se un'informazione \`e elaborata solo da un SI, pi\`u questa sar\`a
  controllata da algoritmi, nell'interazione con l'utente questo accade di meno)
  \item Completezza: se voglio fare un'anagrafica dei dipendenti, mi interessa avere solo certi
  tipi di informazioni (telefono, etc), in funzione degli obiettivi dell'azienda
  \item Precisione: approssimazione dei dati (numeri)
  \item Omogeneit\`a: si deve cercare di riconoscere nel SI tutti i putni in
  cui informazioni di una stessa natura possano essere trattate con stesse funzioni
  \item Fruibilit\`a: facilit\`a con cui l'utente del sistema pu\`o accedere e comprendere
  i dati
\end{itemize}
\subsection{Sistema gestionale classico}
\label{sub:Sistema gestionale classico}
Presenta sistemi specializzati e indipendenti che comunicano tra loro.
Pi\`u il panorama aziendale \`e cos\`i frammentato, pi\`u difficile tenere
traccia dei processi aziendali.
\subsection{Enterprise Resource Planning}
\label{sub:Enterprise Resource Planning}
Sistemi altamente configurabili che permettono un'ampia informatizzazione
della azienda in tutte le aree. Un sistema ERP \`e gestito da un solo fornitore,
\`e altamente configurabile e fornisce una gestione di dati di alt\`a qualit\`a
(nei termini visti precedentemente). Permettono di integrare delle personalizzazioni,
costruendo estensioni del sistema.\\
Il sistema ERP \`e disaccoppiato dal sistema aziendale (ne trascende la struttura).\\
\subsubsection{ERP per l'amministrazione}
\label{subs:ERP per l'amministrazione}
A partire dalla contabilit\`a ordinaria, a seconda del tipo di azienda, l'ERP integra moduli
relativi a contabilit\`a analitica (registrazione ingressi e uscite delle singole parti dell'azienda),
contabilit\`a IVA (solo per aziende private), bduget, controllo di gestione, cespiti e cos\`i via.\\
I diversi moduli presentano interazioni intrinseche con altri moduli
L'ERP e\`usato per trattare determinate informazioni nei vari flussi:
\begin{itemize}
  \item Logistica (movimentazione dei materiali nell'azienda)
  \item Vendite (trattamento delle informazioni commerciali e relative ai clienti)
  \item Acquisti (informazioni sui rapporti coi fornitori)
  \item Produzione (informazioni sulla struttura dei prodotti, delle risorse necessarie
  alla produzioni, dei processi produttivi)
\end{itemize}
Vi sono esigenze informative delle aziende che gli ERP non possono soddisfare.\\
Controllo di qualit\`a, associata a un onere gestionale corposo, il cui volume aumenta
l'attrattiva informatica.\\
Nel settore di ricerca e sviluppo non si seguono processi standardizzabili o definibili
a priori, per cui si possono trovare solo prodotti specializzati come simulatori o CAD,
un minimo supporto informatico gestionale (per monitorare i costi), strumenti di avanzamento
del progetto, gestione della documentazione relativa al progetto e al prodotto. Questo settore
\`e molto destrutturato.\\
Nella manutenzione degli impianti, vi sono sistemi cosiddetti scadenziari (relativi alla pianificazione
degli interventi), sistemi relativi al trattamento dei protocolli e alla gestione dei registri.\\
Poco gettonati invece sono i sistemi relativi alla tesoreria (informazioni
sui rapporti con istituti di credito). Alcuni istituti possono essere pi\`u convenienti di altri,
in base a numerosi parametri e ad algoritmi ad elevata complessit\`a: usati per di pi\`u dalle grandi aziende.\\
Vi sono processi che ampliano le funzionalit\`a di base dell'ERP (ERP II, porta il raggio
d'azione dell'ERP \`e portato dall'interno all'esterno dell'azienda). Alcuni esempi:
\begin{itemize}
  \item CRM (Customer Relationship Management), sostiene tutti i contatti azienda-clienti (specie nella relazione diretta richiesta dal cliente
  come il processo di supporto ai clienti) tra i cui scopi vi \`e la fidelizzazione dei clienti
  nel post-vendita. Ha un'architettura di supporto diversa da quella dell'ERP (ad es. tramite web interface, o mobile app per gli utenti),
  per questo \`e separato dall'ERP.
  \item Supply chain management (SCM), sostiene il rapporto col fornitore tramite internet,
  abbattendo costi e tempi della fornitura.
  \item e-commerce: interfaccia utente che permette di eliminare una funzione aziendale, si integra con l'ERP
  principalmente nel flsso attivo e nella logistica (che pu\`o avere una complessit\`a piuttosto elevata).
  Il sistema si divide in B2C (business to customer) o B2B (business to business).
\end{itemize}
