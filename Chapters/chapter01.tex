% !TEX root = ../Informatica e Aziende.tex
\chapter{Introduzione}
\section{Tipologie di aziende}
Esistono vari tipi di aziende rispetto a determinate caratteristiche.
\subsection{Dimensioni}
\`E possibile definire un'azienda grande o piccola basandosi su due dati numerici:
\begin{itemize}
\item numero di persone che lavorano con l'azienda,
\item quantit\`a di denaro che l'azienda gestisce (\textit{turnover}).
\end{itemize}
\subsection{Topologia}
Grazie alla tecnologia, il luogo del posto di lavoro assume caratteristiche sempre pi\`u complesse.
Vi sono aziende il cui luogo di lavoro \`e
\begin{itemize}
\item singolo (negozio, attivit\`a familiare..)
\item multiplo (aziende con sedi distaccate)
\end{itemize}
Vi sono inoltre aziende cooperanti, ad esempio nella stessa filiera produttiva (Amazon, siti di e-comm..). \\
Il modello cooperative \`e esploso negli ultimi 20 anni (ad es. l'uso dei fornitori nelle industrie automobilistiche).

\subsection{Settori}
Le aziende si differenziano rispetto al settore del mercato in cui lavorano.
\begin{itemize}
\item Industria \\
			Caratterizzata da una produzione continua (si pensi ad un'azienda che produce vino o acciaio) spesso regolamentata da norme vigenti pi\`u o meno stringenti(casa farmaceutiche, ad es.).
\item Commercio \\
			Le merci, gi\`a prodotte altrove, vengono immagazzinate e distribuite ai clienti. L'esempio tipico sono i negozi al dettaglio. In questo caso l'obiettivo \`e quello di massimizzare la velocit\`a con la quali i prodotti vengono distribuiti. \\
Uno dei moderni sistemi di vendita al dettaglio \`e la \textsf{GDO}, grande distribuzione organizzata.
\item Finanza \\
			Aziende che si occupano di gestire il denaro (banche, assicurazioni). Sono pesantemente presenti aspetti formali, strumenti e infrastrutture per la sicurezza e certificazioni.
\item Altro \\
			Qualunque altra aziende che costruisca una infrastruttura su cui fornire dei servizi (Google, Facebook..).
\end{itemize}

\section{Mercato}
Distinguiamo mercati diversi in base a:
\begin{itemize}
\item posto (identificato dai clienti e dalla loro provenienza)(indica, tra le altre cose, le attese di qualit\`a che i clienti hanno)
\item tipo di merci, beni o servizi trattati (alimentare, automobilistico, energetico..)
\item regole che devono essere seguite
\item competizione presente nel mercato
\end{itemize}

\section{Propriet\`a}
Le aziende possono essere di propriet\`a
\begin{itemize}
\item Privata
\begin{enumerate}
\item Singolo proprietario. La sua visione influenza l'intera azienda, lui si occupa un po' di tutto: quando l'azienda cresce \`e difficile trovare qualcuno da delegare, questo pu\`o portare a una crisi.
\item Societ\`a. Vi \`e una maggiore formalizzazione dell'organizzazione con annessa burocrazia (vi sono cio\`e passi formali previsti nel flusso di trasmissione delle informazioni).
\item Multicompany. Un imprenditore gestisce diverse aziende usando una finanziaria ``a cappello" per sostenerle.
\end{enumerate}
\item Pubblica (il proprietario \`e lo stato)
\begin{enumerate}
	\item Servizio pubblico - alta burocrazia
\end{enumerate}
\item Mixed Company. Societ\`a a partecipazione statale.
\item Ad hoc company. Dette anche temporanee, sono aziende create apposta per realizzare un determinato obiettivo, una volta completato, vengono dismesse. Sono previste associazioni temporanee d'impresa.
\end{itemize}

\section{Cosa influenza l'azienda?}
L'influenza \'e bidirezionale. Ogni azienda influenza l'ambiente e ogni azienda \'e influenzata dall'ambiente.
La concorrenza spinge a cambiamenti e innovazioni, che l'azienda \`e rest\`ia a seguire a causa dei costi relativi.
L'apporto dei lavoratori e invisibile, seppur determinante.
Il governo, emanando leggi, influenza le aziende tramite il controllo fiscale e le normative. Spesso le normative hanno lo scopo di diminuire il personale addetto alla gestione della burocrazia.
\section{Cose in comune tra le aziende}
\subsection{Culture}
Nell'aziende vi \`e un insieme di principi dati per assodati dalla maggioranza dei suoi componenti. Tali principi sono:
\begin{itemize}
\item Shared. Principi condivisi implicitamente, difficile da individuare se ci si trova all'interno dell'azienda, ma facilmente riconoscibili se visti da occhi esterni.
\item Determinate persone possono influenzare pi\`u di altre le politiche dall'azienda. Per politiche intendiamo quell'insieme di norme che regolano l'organizzazione informale dell'azienda. 
Quest'ultimo \`e spesso l'antitesi dell'organizzazione informale, che nella vita reale risulta quella determinante.
\end{itemize}
\subsection{Organizzazione}
\begin{itemize}
	\item Infrastruttura, che include sia gli oggetti materiali che non
	\item Procedure interne dell'azienda
	\item Regole, persone che hanno la responsabilit\`a di fare determinate cose
\end{itemize}
\subsection{Livelli}
\begin{itemize}
	\item Operazionali
	\item Gestionali
\end{itemize}
\subsection{Modelli}
\begin{itemize}
	\item Gerarchia, forte e profonda o superficiale e intrecciata
	\item Assegnamento del lavoro
	\item Responsabilit\`a e partecipazione
\end{itemize}
\subsection{Luogo ICT}
\begin{itemize}
	\item Non visibile/Nascosta, dove non c'\`e attribuzione al ruolo o non ci sono persone con le adeguate preparazioni e che hanno la responsabilit\`a dell'ICT.
	\item Esterna, dove ci sono delle persone/aziende esterne che si occupano dell'ICT
	\item Ufficio ICT, dove persone preparate sono in carico della gestione dell'ICT
	\item Divisione ICT (un'espansione dell'ufficio ICT), dove le persone sono organizzate, hanno un certo budget, ma hanno in pi\`u la possibilit\`a di proporre innovazione.
\end{itemize}
