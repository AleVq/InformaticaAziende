% !TEX root = ../Informatica e Aziende.tex
\chapter{Introduzione}
\section{Tipologie di aziende}
Esistono vari tipi di aziende rispetto a determinate caratteristiche.
\subsection{Dimensioni}
\`E possibile definire un'azienda grande o piccola basandosi su due dati numerici:
\begin{itemize}
\item numero di persone che lavorano con l'azienda
\item quantit\`a di denaro che l'azienda gestisce (\textit{turnover})
\end{itemize}
\subsection{Topologia}
Grazie alla tecnologia, il luogo del posto di lavoro assume caratteristiche sempre pi\`u complesse
e permette alle aziende di creare e mantenere efficientemente nuove filiali.
Si viene quindi a formare una distinzione in merito al luogo di lavoro che pu\`o essere:
\begin{itemize}
\item singolo (negozio, attivit\`a familiare..)
\item multiplo (aziende con sedi distaccate)
\end{itemize}
Vi sono inoltre aziende cooperanti, ad esempio nella stessa filiera produttiva (Amazon, siti di e-comm..). \\
Il modello cooperative \`e esploso negli ultimi 20 anni (ad es. l'uso dei fornitori nelle industrie automobilistiche).

\subsection{Settori}
Le aziende si differenziano rispetto al settore del mercato in cui lavorano.
\begin{itemize}
\item Industria \\
			Caratterizzata da una produzione continua (si pensi ad un'azienda che produce vino o acciaio) spesso regolamentata da norme vigenti pi\`u o meno stringenti (ad es. nel caso delle case farmaceutiche vi saranno regolamenti severi).
\item Commercio \\
			Le merci, gi\`a prodotte altrove, vengono immagazzinate e distribuite ai clien\-ti. L'esempio tipico sono i negozi al dettaglio. In questo caso l'obiettivo \`e quello di massimizzare la velocit\`a con la quali i prodotti vengono distribuiti. \\
Uno dei moderni sistemi di vendita al dettaglio \`e la \textsf{GDO}, grande distribuzione organizzata.
\item Finanza \\
			Aziende che si occupano di gestire il denaro (banche, assicurazioni). Sono pesantemente presenti aspetti formali, strumenti e infrastrutture per la sicurezza e certificazioni.
\item Altro \\
			Qualunque altra aziende che costruisca una infrastruttura su cui fornire dei servizi (Google, Facebook..).
\end{itemize}

\section{Mercato}
Distinguiamo mercati diversi in base a:
\begin{itemize}
\item posto, identificato dai clienti e dalla loro provenienza e dal quale \`e possibile dedurre, tra le altre cose, le attese di qualit\`a che i clienti hanno sui prodotti dell'azienda
\item tipo di merci, beni o servizi trattati (alimentare, automobilistico, energetico..)
\item regole che devono essere seguite
\item competizione presente nel mercato
\end{itemize}

\section{Propriet\`a}
Le aziende possono essere di propriet\`a
\begin{itemize}
\item Privata
\begin{enumerate}
\item Singolo proprietario. \\ La sua visione influenza l'intera azienda, lui si occupa un po' di tutto: quando l'azienda cresce \`e difficile trovare qualcuno da delegare, questo pu\`o portare a una crisi.
\item Societ\`a. \\ Vi \`e una maggiore formalizzazione dell'organizzazione con annessa burocrazia (vi sono cio\`e passi formali previsti nel flusso di trasmissione delle informazioni).
\item Multicompany. \\ Un imprenditore gestisce diverse aziende usando una finanziaria \textbf{a cappello} per sostenerle.
\end{enumerate}
\item Pubblica \\ Lo stato \`e considerato il proprietario dell'azienda. Nei servizi pubblici vi \`e un alto livello di burocrazia.
\item Mixed Company. Societ\`a a partecipazione statale.
\item Ad hoc company. Dette anche temporanee, sono aziende create apposta per realizzare un determinato obiettivo. Una volta completato, vengono dismesse. Sono previste associazioni temporanee d'impresa.
\end{itemize}

\section{Cosa influenza l'azienda?}
Nelle aziende l'influenza \`e bidirezionale: ogni azienda influenza l'ambiente e ogni azienda \`e influenzata dall'ambiente.
La concorrenza spinge a cambiamenti e innovazioni, che l'azienda \`e rest\`ia a seguire a causa dei costi associati.
L'apporto dei lavoratori \`e invisibile, seppur determinante.
Il governo, emanando leggi, influenza le aziende tramite il controllo fiscale e le normative. Spesso le normative hanno lo scopo di diminuire il personale addetto alla gestione della burocrazia (tramite l'automazione di processi burocratici o l'implementazione di nuove tecnologie).
\section{Punti comuni tra le aziende}
\subsection{Culture}
Nelle aziende vi \`e un insieme di principi dati per assodati dalla maggioranza dei suoi componenti. Tali principi sono:
\begin{itemize}
\item Shared. Principi condivisi implicitamente, difficile da individuare se ci si trova all'interno dell'azienda, ma facilmente riconoscibili se visti da occhi esterni.
\item Determinati da persone capaci di influenzare pi\`u di altre le politiche dall'azienda. Per politiche intendiamo quell'insieme di norme che regolano l'organizzazione informale dell'azienda.
Quest'ultimo \`e spesso l'antitesi dell'organizzazione informale, che nella vita reale risulta quella determinante.
\end{itemize}
\section{Organizzazione}
L'organizzazione di un'azienda prevede:
\begin{itemize}
	\item Infrastruttura, che include sia gli oggetti materiali che non
	\item Procedure interne dell'azienda
	\item Regole: le persone hanno la responsabilit\`a di fare determinate cose
\end{itemize}
\subsection{Livelli}
\begin{itemize}
	\item Operazionali: relativo all'operativit\`a dell'azienda
	\item Gestionali: relativo ai processi di gestione dell'azienda
\end{itemize}
\subsection{Modelli}
Aziende diverse possono avere modelli diversi nella gestione delle risorse umane, dei compiti e delle responsabilit\`a da assegnare. Vi posso quindi essere gerarchie forti e profonde (come il modello classico piramidale) o superficiali e intrecciate (che vanno diffondendosi sempre pi\`u negli ultimi tempi). Altri tratti caratterizanti l'azienda sono i metodi con cui vengono gestiti:
\begin{itemize}
	\item l'assegnamento del lavoro
	\item la esponsabilit\`a e la partecipazione del personale
\end{itemize}
\subsection{Luogo ICT}
La gestione e l'assegnazione della responsabilit\`a delle tecnologie ICT in un'azienda possono essere:
\begin{itemize}
	\item Non visibili/nascoste, laddove non c'\`e attribuzione al ruolo o non ci sono persone con le adeguate preparazioni e che hanno la responsabilit\`a dell'ICT.
	\item Esterne, dove ci sono delle persone/aziende esterne che si occupano dell'ICT
	\item Interne: \`e previsto un ufficio ICT, dove persone preparate sono in carico della gestione dell'ICT, oppure una divisione ICT (un'espansione dell'ufficio ICT), dove le persone sono organizzate, hanno un certo budget, ma hanno in pi\`u la possibilit\`a di proporre innovazione.
\end{itemize}
