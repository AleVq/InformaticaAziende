% !TEX root = ../Informatica e Aziende.tex
\chapter{Etica}

\section{Riservatezza}
La riservatezza regola parte delle relazioni che intercorrono tra l'azienda e le persone che entrano in relazione con i suoi processi.\\
Al fine di sancire il diritto alla protezione dei dati personali, sono state emanate leggi e regolamenti sia a livello nazionale che europeo:
\begin{itemize}
	\item a livello italiano il decreto legislativo 196 del 2003, detto \bf codice in materia di protezione dei dati personali\rm, coordina in un testo unico tutte le precedenti disposizioni relative alla protezione dei dati personali
	\item a livello sovranazionale, nel 12 marzo 2014 il parlamento europeo ha adottato il \bf pacchetto legislativo sul trattamento dei dati personali \rm che propone regolamenti e direttive (sia nel settore pubblico che in quello privato) relativi alla tutela delle persone fisiche con riguardo al trattamento dei dati personali da parte delle autorit\`a competenti ai fini di prevenzione, indagine, accertamento e perseguimento di reati o esecuzione di sanzioni penali, e la libera circolazione di tali dati
\end{itemize}

\subsection{Principi cardine}
Il problema della riservatezza coinvolge qualunque tipo di azienda e concerne il trattamento di dati anagrafici e di riconoscimento.\\
Si vogliono sanzionare penalmente l'omissione delle misure minime di protezione dei dati e il loro trattamento illecito.\\
Le sanzioni amministrative sono invece destinate a chi omette o presenta in modo non idoneo l'informativa sul trattamento dei dati.\\

\subsection{Principali attori}
Nel trattamento dei dati personali gli attori si possono suddividere in cinque categorie principali:
\begin{itemize}
	\item soggetto interessato: le persone di cui si vogliono trattare i dati
	\item titolare del trattamento: persona fisica o giuridica cui competono le decisioni in merito
all’utilizzo dei dati
	\item responsabile del trattamento: persona fisica o giuridica incaricata dal titolare al trattamento dei dati
	\item incaricato del trattamento: persona fisica incaricata dal titolare o dal responsabile a trattare i dati personali
	\item garante della privacy: l'autorit\`a pubblica che ha l'onere di verificare il coretto trattamento dei dati
\end{itemize}

\subsection{Politiche generali}
In generale vanno adottate le seguenti politiche:
\begin{itemize}
	\item L'informativa sul trattamento dei dati \`e sempre dovuta e pu\`o essere fornita anche solo verbalmente
	\item Nel caso di trattamento di dati sensibili deve sempre essere ottenuto il consenso preventivo dell’interessato
	\item Vanno adottate alcune misure minime di sicurezza (sistemi di autenticazione, gestione delle credenziali, misure a tutela del SI, ad es. antivirus, procedure di back-up e redazione del \bf documento programmatico della sicurezza \rm in caso di dati sensibili o giudiziari)
\end{itemize}

\subsection{Regolamento europeo}
Impone una riforma della privacy agli stati membri entro il 2018.\\
L'obiettivo \`e quello di rendere uniforme la normativa e garantire una pi\`u efficace tutela dei personali al fine di stimolare la creazione di un mercato unico digitale europeo.

\subsection{Principi cardine}
\begin{itemize}
	\item Tutti coloro che trattano dati relativi a cittadini europei sono tenuti a rispettare questo regolamento
	\item Sono oggetto del regolamento tutti i dati personali acquisiti sia esplicitamente che implicitamente (ad es. tramite cookies, indirizzi IP e tag RFID)
	\item Sono previste sanzioni amministrative fino a 100 milioni di euro e fino al 5\% del fatturato
	\item \`E previsto un \bf obbligo di documentazione \rm per la gestione della privacy contenente tutti gli atti e i regolamenti aggiornati, in particolare, il titolare del trattamento deve conservare tutta la documentazione (principio di \bf rendicontazione\rm)
	\item I soggetti interessati hanno diritto all'oblio, cio\`e alla cancellazione totale dei dati e rinuncia alla loro ulteriore diffusione
	\item I soggetti interessati hanno diritto alla portabilità del dato: diritto a trasferire i propri dati, ottenendoli in formato elettronico strutturato di uso comune
\end{itemize}
Inoltre, le aziende sono tenute a:
\begin{itemize}
	\item Adottare politiche concise, chiare (uso di un liguaggio semplice e chiaro) nel informare i soggetti interessati
	\item Effettuare una valutazione dell'impatto dei processi aziendali sulla privacy
	\item Designare un \it data protection officer \rm (DPO), che si occupa nelle grandi aziende di tutte le questioni inerenti il trattamento dei dati
	\item Adottare le politiche di \it privacy by design \rm e \it protection by default\rm: fin dalla progettazione le aziende devono mettersi nell'ottica di gestire la privacy e proteggere i dati personali 
\end{itemize}

\section{Gestire i dilemmi etici in azienda}
Bisogna puntare all'autoregolazione tramite strumenti sociali: dalla
formazione, definizione di macropolitiche (linee guida dettata
dall'amministrazione centrale dell'azienda la cui valenza \`e alla stregua
di una normativa nell'azienda), definizioni di certificazione da parte
delle aziende in tema etico (che possano fondare degli standard),
sensibilizzazione sociale (ha un carattere molto forte, ad es. tramite
pressioni sociali)\\
Nell'azienda vengono fatte delle assunzioni di responsabilit\`a sulle
conseguenze delle proprie azioni. Quando \`e presente una normativa le responsabilit\`a individuali sono ben definite.\\
Intorno alle decisioni ci si pone dei
processi di tipo etico. Il problema etico implica un dilemma
(riconoscere che la conseguenza delle mie azioni pu\'o avere risvolti
negativi su altri) di cui si ricerca una soluzione ottimale attraverso
determinati passi che creano l'\textbf{analisi etica}:

\begin{itemize}

\item
  Identificato il dilemma, si vogliono trovare gli elementi e i fatti
  che lo circondano.\\
\item
  Si definiscono i punti di conflitto e si identificano i valori di
  ordine superiore che diano un \textbf{cappello} di legittimit\`a sociale del
  proprio agire (quali sono i valori che sostengono la mia azienda? come
  fa il pubblico a considerare eticamente corretta la mia azienda).\\
\item
  Si identificano poi gli interessati, coloro che sono coinvolti nel mio
  dilemma etico\\
\item
  Identificazione delle opzioni che si possono adottare per contenere le
  potenziali conseguenze
\end{itemize}

Per facilitare la conduzione di una analisi etica, vi sono vari
principi:

\begin{itemize}

\item
  Mettersi nei panni degli altri
\item
  Correttezza: un'azione non corretta anche solo per uno allora non \`e
  corretta per nessuno (bisogna quindi focalizzarsi anche sui casi
  singoli)
\item
  Ripetibilit\`a: un'azione che non si pu\'o ripetere non va fatta
\item
  Massimizzazione del valore (monetario ma anche sociale, relazionale,
  che non sono tangibili o calcolabili)
\item
  Minimizzare i rischi per aziende o persone
\item
  Propriet\`a: \textbf{non esiste un pranzo gratis} (qualcuno ci ha lavorato
  dietro e vale anche per beni immateriali come il software)
\end{itemize}

Aspetti toccati nell'ICT:

\begin{itemize}

\item
  Privacy
\item
  Propriet\`a intellettuale (segreti commerciali in qualit\`a di
  \textbf{gentlement agreement} tra aziende, copyright come legge che si
  occupa di copie illecite di prodotti materiali o non, brevetti come
  forma pi\'u forte di tutela della propriet\`a intellettuale che include un
  impegno economico non indifferente e in campo del software prevedono
  iter molto lunghi, usati come strumenti di tutela della propriet\`a
  intellettuale)
\item
  Qualit\`a dei SI (qualit\`a dati ed errori dei SI)
\item
  Responsabilit\`a sociale e giuridica
\end{itemize}

Problemi di qualit\`a della vita:

\begin{itemize}

\item
  Rapidit\`a dei cambiamenti
  
  \begin{itemize}
    \item 
      Riduzione dei tempi di riposta alla competizione
    \item 
      Minori ammortizzatori per gli investimenti
    \item
      Just-in-time diffuso
  \end{itemize}
\item
  Continuit\`a della connessione
  \begin{itemize}
    
    \item 
      Il lavoro ti segue
    \item 
      Senso distorto di relazione
    \item 
      Crollo del confine privato/pubblico
  \end{itemize}
\item
  Dipendenza e vulnerabilit\`a
\end{itemize}

